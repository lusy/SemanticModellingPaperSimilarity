\section{Einleitung}

\subsection{Motivation}
lala

% themabeschreibung
% forschungszusammenhang: was gibts schon zum thema, was ist das neue
% und wir wollen das machen, weil...

\subsection{Zielsetzung}
% define keywords/schlüsselkonzepte
% Datensatz
% Verfahren
% benutzte Werkzeuge
% Was will ich damit erreichen/ Welche Ergebnisse strebe ich an?

\subsection{Aufbau der Arbeit}

Im Weiteren wird diese Arbeit folgendermaßen aufgebaut:
\\
Kapitel 2 stellt die theoretischen Grundlagen der Bibliometrie vor, beschreibt semantische Netzwerke als eine Möglichkeit der Wissensrepräsentation und stellt Zusammenhänge zu schon vorhandenen Studien her.
In Kapitel 3 betrachte ich die Eigenschaften und Besonderheiten vom zmath-Datensatz, das modellierte semantische Netz, sowie die Abbildung der Metadate auf das entworfene Schema und definiere ein Ähnlichkeitsmaß für mathematische Publikationen.
In Kapitel 4 untersuche ich den zugrundeliegenden Datensatz auf Ähnlichkeiten mit Hilfe des bereits erarbeiteten Ähnlichkeitsmaßes und stelle eine detaillierte Auswertung der erzielten Ergebnisse dar.
Kapitel 5 gibt einen Ausblick und schlägt Ansätze für weiterführende Untersuchungen vor und fasst nochmal abschließend die Arbeit zusammen.

%-present the subject, it could be with an example
%-define the important words
%-present the hypothesis + arguments etc.
%-describe how the body is organized

%Quite literally, the Introduction must answer the questions, "What was I studying? Why was it an important question? What did we know about it before I did this study? How will this study advance our knowledge?"

