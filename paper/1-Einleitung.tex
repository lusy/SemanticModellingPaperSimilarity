\section{Einleitung}

\subsection{Motivation}
Titel, Thema, Autor, Erscheinungsjahr, Journal und Literaturverweise sind nur einige f�r eine wissenschaftliche Publikation relevante Metadaten.
Diese stehen in bestimmten Relationen zueinander.
Es gibt schon l�nger mehrere Repr�sentationen von wissenschaftlichen Arbeiten: Zitations- oder Koautorgraphen ber�cksichtigen aber beispielsweise nur Referenzen oder Autoren.
Zus�tzliche Metadaten k�nnen jedoch f�r eine Analyse ebenfalls bedeutende und interessante Einsichten bringen.
Durch Aufbau eines semantischen Netzes, das alle zu einer Publikation relevanten Informationen enth�lt, soll ein allgemeineres Modell f�r wissenschaftliche Publikationen entwickelt werden.
Durch die Ber�cksichtigung der in diesem Modell erfassten semantischen Relationen wird dann versucht ein m�glichts pr�zises �hnlichkeitsma� f�r wissenschaftliche Publikationen zu entwerfen. %vlt beispiele (A ist Autor von Paper P, P ist Journal C erschienen, hat X,Y,Z als Keywords usw)
Ein gutes �hnlichkeitsma� findet mehrere Gebr�uche: es kann unterm anderen bei automatischer Klassifizierung von Arbeiten in einer wissenschaftlichen Datenbank oder als Ausgangspunkt f�r das Vorschlagen verwandten/relevanten Dokumente eingesetzt werden.
Fr�here Entw�rfe von �hnlichkeitsma�en beziehen nur bibliometrische Kategorien wie Zitationsanalysen (von direkten Zitaten, Kozitationen oder bibliographischen Kopplungen) oder Koautorenschaften ein.
Indem alle Metadaten zu einer Publikation und ihre semantische Erfassung in eine neue Definition integriert werden, sollen genauere Ergebnisse erzielt werden.


% themabeschreibung (vlt mit beispiel)
% und wir wollen das machen, weil... (warum ist die frage wichtig/interessant)
% forschungszusammenhang: was gibts schon zum thema, was ist das neue


\subsection{Zielsetzung}
% define keywords/schl�sselkonzepte
% Was will ich damit erreichen/ Welche Ergebnisse strebe ich an?
% Datensatz
% Verfahren
% benutzte Werkzeuge

Ziel dieser Arbeit ist es ein semantisches Netz zu modellieren, das alle zu einer mathematischen Publikation relevanten Metadaten ber�cksichtigt und mathematische Publikationen zueinander in Relation stellt.
Im zweiten Teil der Arbeit wird dieses Netzwerk als Grundlage f�r die Formulierung eines �hnlichkeitsma�es und �hnlichkeitsanalysen verwendet.
Die Qualit�t des definierten �hnlichkeitsma�es wird ermittelt, indem die Ergebnisse, die es liefert, mit der schon im Datensatz durch Fachexperten vergebenen MSC-Klassifizierung
\footnote{Die Mathematical Subject Classification (MSC) ist eine Klassifizierungskonvention, die durch die wissenschaftlichen Datenbanken Mathematical Reviews und das Zentralblatt Mathematik genutzt wird. F�r den Aufbau siehe http://www.ams.org/mathscinet/msc/msc2010.html} verglichen werden.
\\
Der der Ausarbeitung zugrundeliegende Datensatz beinhaltet die Metadaten und MSC-Klassifikationen von ca. 750 000 Publikationen vom Zentralblatt Mathematik, FIZ Karlsruhe.
Diese werden mit Hilfe von Ontology-Engineering-Tools und einem in $Python$ geschriebenen Parser auf ein semantisches Netz abgebildet, das in der Sprache $OWL$ modelliert ist.
% wie wird das �hnlichkeitsma� definiert/evaluiert (Verfahren + Werkzeuge)

\subsection{Aufbau der Arbeit}

Im Weiteren wird diese Arbeit folgenderma�en aufgebaut:
\\
Kapitel 2 stellt die theoretischen Grundlagen der Bibliometrie vor, beschreibt semantische Netzwerke als eine M�glichkeit der Wissensrepr�sentation und stellt Zusammenh�nge zu schon vorhandenen Studien her.
In Kapitel 3 betrachte ich die Eigenschaften und Besonderheiten vom zmath-Datensatz, das modellierte semantische Netz, sowie die Abbildung der Metadate auf das entworfene Schema und definiere ein �hnlichkeitsma� f�r mathematische Publikationen.
In Kapitel 4 untersuche ich den zugrundeliegenden Datensatz auf �hnlichkeiten mit Hilfe des bereits erarbeiteten �hnlichkeitsma�es und stelle eine detaillierte Auswertung der erzielten Ergebnisse dar.
Kapitel 5 gibt einen Ausblick und schl�gt Ans�tze f�r weiterf�hrende Untersuchungen vor und fasst nochmal abschlie�end die Arbeit zusammen.

%-present the subject, it could be with an example
%-define the important words
%-present the hypothesis + arguments etc.
%-describe how the body is organized

%Quite literally, the Introduction must answer the questions, "What was I studying? Why was it an important question? What did we know about it before I did this study? How will this study advance our knowledge?"

