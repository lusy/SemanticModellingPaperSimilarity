\section*{Appendix}

% Um ein Gefühl dafür zu bekommen, wie der Algorithmus funktioniert, werden hier die ersten Schritte, die für die Ähnlichkeitsberechnung folgender zwei Publikationen aus dem \textit{zmath}-Datensatz notwendig sind, ausgeführt.
% Publikationen: 
%:id:    3006698
%:an:    0005.25102
%:py:    1931
%:au:    Myrberg, P.J.
%:ai:    myrberg.pekka-j
%:ti:    \"Uber beschr\"ankte Funktionen in mehrfach zusammenh\"angenden Bereichen.
%:so:    Ann. Acad. Sci. Fenn., Ser. A 33, No.8, 1-15 (1931).
%:cc:    
%:ut:    complex functions
%:la:    DE
%:ci:    
%:li:    
%::::
%:::
%:id:    5166522
%:an:    1153.00002
%:py:    2005
%:au:    Leonov, Sergey A.; Leonov, Alexander I.
%:ai:    leonov.sergey-a; leonov.alexander-i
%:ti:    Mathematical handbook for electrical engineers.
%:so:    Artech House Technology Management and Professional Development Library. London: Artech House (ISBN 978-1-58053-779-7/hbk). xviii, 495~p. \sterling~85.00 (2005).
%:cc:    00A06 26-00 35-00 60-00 62-00 78-00 94-00
%:ut:    algebra; functions; equations; combinations; planar and solid geometry; trigonometry; analytic geometry; integral and differential calculus; differential equations; complex functions; Fourier series; vector algebra; probability; applied statistics; computer-aided computation; electrical circuits; antennas; waves; scattering; signal processing; stochastic radio engineering
%:la:    EN
%:ci:    
%:li:    


\begin{math}
\begin{array}{lcl}
R_1(a,b) & = & 
        \lambda_1\times\cfrac{c}{1*21}\times(R_0(K_1(a),K_1(b)) + R_0(K_1(a),K_2(b))+... + R_0(K_1(a),K_{21}(b)))
        \\ & + &
        \lambda_2\times\cfrac{c}{1*2}\times(R_0(A_1(a),A_1()) + R_0(A_1(a),A_2(b)))
        \\ & + &
        \lambda_3\times\cfrac{c}{1*1}\times R_0(S_1(a),S_1(b))
        \\ & + &
        0
        \\ & + &
        \lambda_5\times\cfrac{c}{1*1}\times R_0(Y_1(a),Y_1(b))
        \\ & = &

\end{array}
\end{math}



\[
\begin{array}{lcl}
 R_{k+1}(a,b) & = & 
        \lambda_1\times\cfrac{c}{|K(a)||K(b)|}
        \sum_{i=1}^{|K(a)|} \sum_{j=1}^{|K(b)|} R_k(K_i(a),K_j(b))
        \\ & + &
        \lambda_2\times\cfrac{c}{|A(a)||A(b)|}
        \sum_{i=1}^{|A(a)|} \sum_{j=1}^{|A(b)|} R_k(A_i(a),A_j(b))
        \\ & + &
        \lambda_3\times\cfrac{c}{|S(a)||S(b)|}
        \sum_{i=1}^{|S(a)|} \sum_{j=1}^{|S(b)|} R_k(S_i(a),S_j(b))
        \\ & + &
        \lambda_4\times\cfrac{c}{|C(a)||C(b)|}
        \sum_{i=1}^{|C(a)|} \sum_{j=1}^{|C(b)|} R_k(C_i(a),C_j(b))
        \\ & + &
        \lambda_5\times\cfrac{c}{|Y(a)||Y(b)|}
        \sum_{i=1}^{|Y(a)|} \sum_{j=1}^{|Y(b)|} R_k(Y_i(a),Y_j(b))
\end{array}
\]
\newline
\text{wenn a und b aus sonstiger Klasse:}
\[
R_{k+1}(a,b)  = 
        \cfrac{c}{|L(a)||L(b)|}
        \sum_{i=1}^{|L(a)|} \sum_{j=1}^{|L(b)|} R_k(L_i(a),L_j(b))
\]


