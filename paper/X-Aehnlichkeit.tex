\section{�hnlichkeitsma�}

In Anlehnung an den Arbeiten von Jeh und Widom \cite{simrank2002}, Zhao, Han und Sun \cite{ZhaoHS09} und Yoon, Kim und Park \cite{DBLP:journals/corr/abs-1109-1059}
wird das folgende rekursive �hnlichkeitsma�, das semantische Relationen zwischen den Metadaten von wissenschaftlichen Publikationen ber�cksichtigt, vorgeschlagen.

%Ma�definition
\newtheorem{mydef}{Semantische �hnlichkeit}
\begin{mydef}
\[
 R_0(a,b) =
    \begin{cases}
     1 & \text{wenn } $a$ = $b$ \\
     0 & \text{sonst}\\
    \end{cases}
\]
\[
\begin{array}{lcl}
 R_{k+1}(a,b) & = & 
        \lambda_1\times\cfrac{c}{|K(a)||K(b)|}
        \sum_{i=1}^{|K(a)|} \sum_{j=1}^{|K(b)|} R_k(K_i(a),K_j(b))
        \\ & + &
        \lambda_2\times\cfrac{c}{|A(a)||A(b)|}
        \sum_{i=1}^{|A(a)|} \sum_{j=1}^{|A(b)|} R_k(A_i(a),A_j(b))
        \\ & + &
        \lambda_3\times\cfrac{c}{|S(a)||S(b)|}
        \sum_{i=1}^{|S(a)|} \sum_{j=1}^{|S(b)|} R_k(S_i(a),S_j(b))
        \\ & + &
        \lambda_4\times\cfrac{c}{|C(a)||C(b)|}
        \sum_{i=1}^{|C(a)|} \sum_{j=1}^{|C(b)|} R_k(C_i(a),C_j(b))
        \\ & + &
        \lambda_5\times\cfrac{c}{|Y(a)||Y(b)|}
        \sum_{i=1}^{|Y(a)|} \sum_{j=1}^{|Y(b)|} R_k(Y_i(a),Y_j(b))
\end{array}
\]
\end{mydef}

Das vorgeschlagene Ma� grenzt sich folgenderma�en von fr�heren Arbeiten ab:
\\
\\
SimRank und rvs-SimRank \cite{simrank2002} messen allgemeine strukturelle �hnlichkeit in Graphstrukturen, jeweils auf die ein- bzw. auf die ausgesehenden Kanten des Graphs gest�tzt.
\\
\\
P-Rank \cite{ZhaoHS09} verallgemeinert die zwei Ma�e, indem es sowohl ein- als auch ausgehenden Kanten von Informationsnetzwerken, in einer bestimmten Gewichtung, betrachtet.
\\
\\
C-Rank \cite{DBLP:journals/corr/abs-1109-1059} bezieht sich schon, im Gegensatz zu allen eben erw�hnten Ans�tzen, konkret auf �hnlichkeit zwischen wissenschaftlichen Publikationen. Das Ma� ber�cksichtigt aber nur Zitationen und unterscheidet die Richtung des Zitats nicht (d.h. unterscheidet nicht, ob Paper $A$ Paper $B$ zitiert oder anders rum).


