\section{�hnlichkeitsma�}

In Anlehnung an den Arbeiten von Jeh und Widom \cite{simrank2002}, Zhao, Han und Sun \cite{ZhaoHS09} und Yoon, Kim und Park \cite{DBLP:journals/corr/abs-1109-1059}
wird das folgende rekursive �hnlichkeitsma�, das semantische Relationen zwischen den Metadaten von wissenschaftlichen Publikationen ber�cksichtigt, vorgeschlagen.
Es wird rekursiv die �hnlichkeit zwischen jeden 2 Elementen, die der selben Kategorie angeh�ren, aufgrund von den semantischen Relationen im modellierten semantischen Metadatengraphen f�r wissenschaftliche Publikationen ermittelt: jeden 2 Publikationen, jeden 2 Keywords, jeden 2 Autoren, jeden 2 Quellen und jeden 2 Erscheinungsjahren.
Dabei flie�en bei der �hnlichkeitsberechnung f�r 2 Publikationen die folgenden Relationen ein: (Publikation $hasKeyword$ Keyword), (Publikation $hasAuthor$ Autor), (Publikation $isPublishedIn$ Source), (Publikation $cites$ Publikation) und (Publikation $isCitedBy$ Publikation), (Publikation $wasPublishedInYear$ Erscheinungsjahr).
Bei der �hnlichkeitsberechnung f�r 2 Elemente jeder anderen Kategorie werden nur die Beziehungen zwischen Elementen der entsprechenden Kategorie und Publikationen ber�cksichtigt, da nur solche Beziehungen im vorgeschlagenen semantischen Netz modelliert sind.
%Ma�definition rekursiv
\newtheorem{mydef}{Semantische �hnlichkeit - rekursiv}
\begin{mydef}
\[
sim(a,b) = 1 \quad \text{wenn } a=b
\]
\newline
\text{sonst}
\newline
\text{wenn a und b Publikationen:}
\[
\begin{array}{lcl}
 sim(a,b) & = & 
        \lambda_1\times\cfrac{c}{|K(a)||K(b)|}
        \sum_{i=1}^{|K(a)|} \sum_{j=1}^{|K(b)|} sim(K_i(a),K_j(b))
        \\ & + &
        \lambda_2\times\cfrac{c}{|A(a)||A(b)|}
        \sum_{i=1}^{|A(a)|} \sum_{j=1}^{|A(b)|} sim(A_i(a),A_j(b))
        \\ & + &
        \lambda_3\times\cfrac{c}{|S(a)||S(b)|}
        \sum_{i=1}^{|S(a)|} \sum_{j=1}^{|S(b)|} sim(S_i(a),S_j(b))
        \\ & + &
        \lambda_4\times\cfrac{c}{|C(a)||C(b)|}
        \sum_{i=1}^{|C(a)|} \sum_{j=1}^{|C(b)|} sim(C_i(a),C_j(b))
        \\ & + &
        \lambda_5\times\cfrac{c}{|Y(a)||Y(b)|}
        \sum_{i=1}^{|Y(a)|} \sum_{j=1}^{|Y(b)|} sim(Y_i(a),Y_j(b))
\end{array}
\]
\newline
\text{wenn a und b Elemente einer anderen Kategorie:}
\[
sim(a,b)  = 
        \cfrac{c}{|L(a)||L(b)|}
        \sum_{i=1}^{|L(a)|} \sum_{j=1}^{|L(b)|} sim(L_i(a),L_j(b))
\]
\newline
\newline
\newline
wobei $K$ alle Keyword-Relationen, $A$ alle Autor-Relationen, $S$ alle Source-Relationen, $C$ alle Zitationsrelationen und $Y$ alle Erscheinungsjahrrelationen einer Publikation darstellen.
Dabei wird mit $K(a)$ die Menge aller Keywords von Paper a bezeichnet.
(Die Menge aller Elemente, die in Relation $hasKeyword$ zum Paper a stehen.)
$K_i(a)$ ist dann das i. Keyword von a.
(Analog sind $A(a)$ alle Autoren von Paper a, $A_i(a)$ ist der i. Autor von Paper a u.s.w.)
$L(a)$ und $L(b)$ beschreiben f�r 2 Elemente a und b, die der selben semantischen Kategorie angeh�ren, die Mengen aller Elemente, die mit a, bzw b, in Relation stehen.
\newline
\newline
$\sum_{i=1}^{5} \lambda_i = 1$ sind die Gewichtungen von den verschiedenen Relationen, die die Bedeutung von den einzelnen Metadaten priorisieren; es sollten 2 verschiedene Priorisierungsreihenfolgen bestimmt und ausgewertet werden.
\newline
\newline
$c$ ist ein D�mfungsfaktor, $c\in [0..1]$; c wird in Anlehnung an \cite{simrank2002}, \cite{ZhaoHS09} und \cite{DBLP:journals/corr/abs-1109-1059} bestimmt, die verschiedenen Werte ausprobieren und eine optimale Konstante in dem jeweiligen Fall vorschlagen.
\newline
\end{mydef}

Die rekursive Definition kann folgenderma�en iterativ umgeschrieben werden.
\cite{simrank2002}, \cite{ZhaoHS09} und \cite{DBLP:journals/corr/abs-1109-1059} haben gezeigt, dass die so definierten iterativen Gleichungen zu einem Fixpunkt konvergieren.

%Ma�definition
\newtheorem{mydef2}{Semantische �hnlichkeit - iterativ}
\begin{mydef2}
\[
 R_0(a,b) =
    \begin{cases}
     1 & \text{wenn } $a$ = $b$ \\
     0 & \text{sonst}\\
    \end{cases}
\]
\newline
\text{wenn a und b Publikationen:}
\[
\begin{array}{lcl}
 R_{k+1}(a,b) & = & 
        \lambda_1\times\cfrac{c}{|K(a)||K(b)|}
        \sum_{i=1}^{|K(a)|} \sum_{j=1}^{|K(b)|} R_k(K_i(a),K_j(b))
        \\ & + &
        \lambda_2\times\cfrac{c}{|A(a)||A(b)|}
        \sum_{i=1}^{|A(a)|} \sum_{j=1}^{|A(b)|} R_k(A_i(a),A_j(b))
        \\ & + &
        \lambda_3\times\cfrac{c}{|S(a)||S(b)|}
        \sum_{i=1}^{|S(a)|} \sum_{j=1}^{|S(b)|} R_k(S_i(a),S_j(b))
        \\ & + &
        \lambda_4\times\cfrac{c}{|C(a)||C(b)|}
        \sum_{i=1}^{|C(a)|} \sum_{j=1}^{|C(b)|} R_k(C_i(a),C_j(b))
        \\ & + &
        \lambda_5\times\cfrac{c}{|Y(a)||Y(b)|}
        \sum_{i=1}^{|Y(a)|} \sum_{j=1}^{|Y(b)|} R_k(Y_i(a),Y_j(b))
\end{array}
\]
\newline
\text{wenn a und b Elemente einer anderen Kategorie:}
\[
R_{k+1}(a,b)  = 
        \cfrac{c}{|L(a)||L(b)|}
        \sum_{i=1}^{|L(a)|} \sum_{j=1}^{|L(b)|} R_k(L_i(a),L_j(b))
\]
\end{mydef2}
Das so definierte Ma� hat die Eigenschaften Symmetrie, Monotonie, Existenz und Eindeutigkeit.
Die entsprechenden Beweise k�nnen aus \cite{ZhaoHS09} und \cite{DBLP:journals/corr/abs-1109-1059} entnommen werden.
\newline
\newline
Das vorgeschlagene Ma� grenzt sich folgenderma�en von fr�heren Arbeiten ab:
\\
\\
SimRank und rvs-SimRank \cite{simrank2002} messen allgemeine strukturelle �hnlichkeit in Graphstrukturen, jeweils auf die ein- bzw. auf die ausgehenden Kanten des Graphs gest�tzt.
\\
\\
P-Rank \cite{ZhaoHS09} verallgemeinert die zwei Ma�e, indem es sowohl ein- als auch ausgehenden Kanten von Informationsnetzwerken, in einer bestimmten Gewichtung, betrachtet.
\\
\\
C-Rank \cite{DBLP:journals/corr/abs-1109-1059} bezieht sich schon, im Gegensatz zu allen eben erw�hnten Ans�tzen, konkret auf �hnlichkeit zwischen wissenschaftlichen Publikationen. Das Ma� ber�cksichtigt aber nur Zitationen und unterscheidet die Richtung des Zitats nicht (d.h. unterscheidet nicht, ob Paper $A$ Paper $B$ zitiert oder anders rum).

% Notes on accuracy, decay factor and convergence:
% ------------------------------------------------

% Accuracy: C-Rank > SimRank > P-Rank > rvs-SimRank

% Number of iterations:
% SimRank converges at k = 3,
% rvs-SimRank converges at k = 5,
% P-Rank converges at k = 6,
% C-Rank converges at k = 9.

% Decay factor:
% It is obvious that the similarity score of C-Rank increases with the increase of C.
% When C = 0.2, C-Rank converges fast at k = 2.
% When C = 0.8, on the other hand, C-Rank converges at the 9-th iteration.
% When C is low, the recursive power of C-Rank is weakened such that only the papers in local or near-local neighborhood are used in similarity computation.
% When C is high, more papers in a more global neighborhood can be used in computing the similarity recursively. When C is high, therefore, the convergence takes more time.
