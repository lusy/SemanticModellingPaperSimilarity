\section{Entwurf und Implementierung}

\subsection{Analyse des zugrundeliegenden Datensatzes}
% Erklaerung was das Zentralblatt Mathematik ist - siehe homepage
% Beschreibung wie der Datensatz aufgebaut ist, welche Besonderheiten sind aufgefallen
% Beispiel zeigen
% Beschreibung der Eigenschaften (bisschen statistische Analysen?)
% vlt eine Auflistung davon, welche Daten arbitr�r sind (nicht immer ausgef�llt im Datensatz)

lalalal

\subsection{Modellierung des semantischen Netzes}
% Wie sieht es aus?
% Welche �berlegungen gab es bei der Designentscheidung? Warum hab ich an Stellen, wo mehrere Ans�tze m�glich waren genau das gew�hlt?
% Was k�nnte vlt anders/besser realisiert werden
% Vlt technische Details/Grenzen/Eckdaten (Laufzeit, Speicherplatz, etc)
% Vlt Werkzeuge

\subsection{Mapping auf das entworfene Schema}
% technische Details
% Werkzeuge
% Designentscheidungen (wir Mappen die Sprachen jedes mal mit/doch nicht, ...) begr�nden

\subsection{Ein �hnlichkeitsma� f�r mathematische Publikationen}
% Definieren
% Entscheidungen begr�nden

% Idee von paper \cite{Yang:2009:TRSN} nutzen, um das Ma� zu definieren
