\section{Entwurf und Implementierung}

\subsection{Analyse des zugrundeliegenden Datensatzes}
% Erklaerung was das Zentralblatt Mathematik ist - siehe homepage
% Beschreibung wie der Datensatz aufgebaut ist, welche Besonderheiten sind aufgefallen
% Beispiel zeigen
% Beschreibung der Eigenschaften (bisschen statistische Analysen?)
% vlt eine Auflistung davon, welche Daten arbitr�r sind (nicht immer ausgef�llt im Datensatz)

lalalal
% Der Datensatz beinhaltet die Metadaten zu 2907086 Publikationen

\subsection{Modellierung des semantischen Netzes}
% Wie sieht es aus?
% Welche �berlegungen gab es bei der Designentscheidung? Warum hab ich an Stellen, wo mehrere Ans�tze m�glich waren genau das gew�hlt?
% Was k�nnte vlt anders/besser realisiert werden
% Vlt technische Details/Grenzen/Eckdaten (Laufzeit, Speicherplatz, etc)
% Vlt Werkzeuge
Diese werden mit Hilfe von Ontology-Engineering-Tools und einem in $Python$ geschriebenen Parser auf ein semantisches Netz abgebildet, das in der Sprache $OWL$ modelliert ist.
%

% Mein Modell ist eigentlich ein semantisches Netz, worauf auch graphen-theoretische Methoden angewandt werden

\subsection{Mapping auf das entworfene Schema}
% technische Details
% Werkzeuge
% Designentscheidungen (wir Mappen die Sprachen jedes mal mit/doch nicht, ...) begr�nden

\subsection{Ein �hnlichkeitsma� f�r mathematische Publikationen}
% Definieren
% Entscheidungen begr�nden

% Combine using the semantic network:
% Zitationen (die verschiedenen Auspr�gungen davon auch)
% Autoren
% Keywords - take a look at paper \cite{Yang:2009:TRSN}
% Sources
% Jahren (Vermutung: Paper, die mit einem gro�en zeitlichen Abstand ver�ffentlicht worden sind, sind eher weniger �hnlich; Idee: ausprobieren mit verschiedenen Zeitfenstern, innerhalb davon die Papers �hnlich sein sollten)

% Eine gewichtete Kombination von diesen Relationen
% Begr�ndung: warum ist die Gewichtung so ausgefallen
% 3 verschiedene Varianten machen:
%%% gleichgewichtet: f�r Vergleich; Gleichgewichtung macht in dem Sinne keinen Sinn, weil daf�r br�uchten wir nicht zwischen verschiedenen Relationen unterscheiden
%%% 2 andere - nach Bauchgef�hl
%% aber begr�ndet: z.b. keywords, h�here gewichtung wegen \cite{Yang:2009:TRSN}, bei publikationsjahren: zeitfenster, literatur veraltet \cite{frank2009einfuehrung}
% take a look at how to compute sim rank: consider the bipartite variant - what is the difference between In-Neighbours and Out-Neighbours in my case? Should I make one?


% Idee von paper \cite{Yang:2009:TRSN} nutzen, um das Ma� zu definieren
