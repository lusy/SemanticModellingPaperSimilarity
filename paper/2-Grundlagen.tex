\section{Grundlagen}
% Vlt Reihenfolge von Bibliometrie und Semantische Netzwerke umdrehen

\subsection{Bibliometrie}
% Ganz kurz was ist die Bibliometrie, Forschungsschwerpunkt
% Was sind wichtige Parameter/Kenngrößen, die für uns relevant sind/wir verwenden
% was sind die Kritikpunkte dran/ Warum glaube ich, dass diese Kenngrößen alleine nicht genug sind, um Ähnlichkeit zu definieren


% Was ist Bibliometrie?:
% Problemstellung/Methoden oder Kombination von beiden müssen bei einer wissenschaftlichen Arbeit neu sein (damit sie einen Wert hat)
% Die Bibliometrie bietet nachweisbare adäquate Methoden, um diese Problemstellung zu beantworten \cite{frank2009einfuehrung}
% guck weiter im anderen Bibliometriepaper (88 seiten)


% mit Bildern
% Bibliographische Kopplung:
% Def: 2 Publikationen zitieren ein gemeinsames früheres Werk
% dadurch werden Beziehungen zwischen frisch publizierten Arbeiten her-/festgestellt; oder von Artikeln, die in einem geringen zeitlichen Abstand von einander publiziert worden sind
% Problem für Ähnlichkeit: 2 Arbeiten, die eine 3. zitieren, können sich womöglich auf komplementäre Aspekte davon beziehen
% Problem Nr 2: Standardwerke und frühere Werke werden öfter zitiert
% Problem Nr 3: Es gibt viele Selbstzitationen, die bestimmte Parameter (Rankings auf Grund von Zitationszahlen/etc) hochpuschen wollen
% Problem Nr 4: Artikel mit langen Referenzlisten sind tendenziell stärker bibliographisch gekoppelt als solche mit kurzen Listen, wenn man nur die Anzahl der zusammen zitierten Arbeiten zählt - Ansatz zur Normalisierung relative bibliographische Kopplung (Jaccard- und Salton-Index), messen die Kopplung bezogen auf die Länge der Referenzlisten


% Kozitation:
% Def: 2 Publikationen werden gemeinsam in einer dritten zitiert
% "Clusters of highly co-cited documents are considered to have high mutual dependance"\cite{DBLP:conf/amcis/KimB08}

% Direktes Zitat
% Problem: Dankbarkeitszitate, Selbstzitate, Leute, die sich (persönlich) kennen, tendieren dazu sich auch öfters zu zitieren gegenseitig

% Koautorenschaft:
% Def:

% -> Auf Grund von den 3 Begriffen kann man Zitationsnetzwerke von Artikeln bauen (de Solla Price, 1965) und Ähnlichkeit über Abstände in so einem Netzwerk definieren.

% Pearsons Korrelationskoeffizient (wird auch oft für Ähnlichkeitsmessungen verwendet)

% Probleme von bibliometrischen Verfahren als Ähnlichkeitsindikatore, zusammengefasst:
% Selbstzitate (Leute pushen ihre Parameter hoch/ Druck von Herausgebern von Journalen, dass Zitate auf dem Journal gemacht werden)
% "Dankbarkeitszitate"
% Zitate, die andere Arbeiten referenzieren ohne Konsultieren des Originalwerks (z.b. bei Quellen von meinen Quellen übernommen)
% Rechtschreibfehler in Zitate - das zitierte Werk kann nicht eindeutig bestimmt/wiedergefundnen werden
% Zitate von Standardwerken/Reviewartikel sind oft
% Frühere Werke werden öfters zitiert
% Manchmal werden die Zitate weggelassen
% Es dauert, bis sich eine Arbeit etabliert hat, und es angefangen wird, dass sie zitiert wird - Ähnlichkeit kann nicht nur auf Grund von Zitationen festgestellt werden

% Probleme wenn (Ko-)Autoren als Ähnlichkeitskenngröße genutzt werden:
% Es wird oftmals nur der 1. Autor pro Paper berücksichtigt - verzerrtes Bild, manche erhalten mehr Anerkennung, etc (wobei neuere Analysen den Beitragsanteil eines Autors zu einem Paper bestimmen, in dem sie durch die Anzahl der Koautoren teilen)
% Bei langen Autorenlisten sind manche Autorinnen nur teilweise im Thema involviert
% Der selbe Autor(in) taucht oft unter verschiedenen Namen auf (Martin, J.; Martin, J.M.; Martin, James); einfaches Stringmatching scheitert
% Ein oft vorkommende Name, der sich auf verschiedene Personen bezieht, kann versehntlich aggregiert werden (Smith)
% -> Lösung für die letzten 2: Benutzen eines eindeutigen Identifiers für jeden Autor (unser Datensatz hat das)

lalala

\subsection{Semantische Netzwerke}
% Was sind semantische Netzwerke
% Warum sind sie in unserem Fall wichtig?/Warum haben wir uns für so eine Wissensrepräsentation entschieden?
% (Wissensrepräsentationen sollen die folgenden Eigenschaften haben: eindeutig sein; klar und konsistent sein; adäquat für die Ziele, für die wir sie brauchen, sein; Rechnen drauf soll möglich sein/sollen berechenbar sein

% das und das unten kommt aus \cite{Bench-Capon:1990:KR}


% Entitäten als Knoten
% Relationen zwischen denen als Kanten
% Klassen vs Individuen
% Klassen und Individuen können auch Attribute haben
% Charakteristik: die Informationen zu einem bestimmten Objekt werden in der Nähe von diesem Objekt geclustert

% und außerdem (für Ontologien):
% Restriktionen
% Regeln
% Axiome
% Events

\subsection{Verwandte Arbeiten}
% Was gibt es schon zum Thema semantische Modellierung von Publikationen (Science Ontology, Diplomarbeit vom einen Typen)
% Was gibt es schon zum Thema Ähnlichkeitsmaßen (Kozitationen, Koautorenschaften, bibliographische Kopplung,

% Ähnlichkeit zwischen Autoren: \cite{Yang:2009:TRSN})
% Ziel vom Paper: Modellierung eines sozialen Netzwerkes von Autoren
% Haben mehrere Methoden genutzt und mit einander verglichen: anhand von Schlüsselwörtern/Keywords, anhand von den persönlichen Interessen der Autoren (von den Autoren selber bestimmt/aus deren Webpages extrahiert), anhand von Themen der Konferenzen, wo Autoren hingehen/Papers veröffentlichen, anhand von Koautorenschaften;
% Die Methoden wurden sowohl einzeln angewandt als auch gewichtet zusammen
% Keywords basierte Graphen wurden von den Autoren selber als aussagekräftiger gewertet als solche, die auf Koautorenschaften basierten
% Probleme, wenn man nur Koautoren nutzt (siehe oben in der Beschreibung von Bibliometrie)
% Probleme, wenn man Keywords nutzt: das selbe Keyword kann unterschiedlich geschrieben werden (web2.0 vs Web 2.0); hat den Nachteil, dass die Untersuchten Papers auf der selben Sprache sein sollten (bei uns sind English Keywords im Datensatz definiert, es kann aber nicht immer auf die Häufigkeit dieser im Volltext zurückgegriffen werden; wobei das werde ich vermutlich auch gar nicht tun)


% "Information on individual journal citations can be used to examine the impact of journals on the subject field and a map of co-cited documents can be used to identify subject specialties and sub-specialties"\cite{DBLP:conf/amcis/KimB08}
% ebenso da: intercitation analysis (between clusters) vs co-citation analysis (within a cluster) - can be used to complement each other and fill in each others weaknesses

