\section{Grundlagen}
% Vlt Reihenfolge von Bibliometrie und Semantische Netzwerke umdrehen

\subsection{Bibliometrie}
% Ganz kurz was ist die Bibliometrie, Forschungsschwerpunkt
% Was sind wichtige Parameter/Kenngrößen, die für uns relevant sind/wir verwenden
% was sind die Kritikpunkte dran/ Warum glaube ich, dass diese Kenngrößen alleine nicht genug sind, um Ähnlichkeit zu definieren

% mit Bildern
% Bibliographische Kopplung:
% Def: 2 Publikationen zitieren ein gemeinsames früheres Werk
% dadurch werden Beziehungen zwischen frisch publizierten Arbeiten her-/festgestellt; oder von Artikeln, die in einem geringen zeitlichen Abstand von einander publiziert worden sind
% Problem für Ähnlichkeit: 2 Arbeiten, die eine 3. zitieren, können sich womöglich auf komplementäre Aspekte davon beziehen
% Problem Nr 2: Standardwerke und frühere Werke werden öfter zitiert
% Problem Nr 3: Es gibt viele Selbstzitationen, die bestimmte Parameter (Rankings auf Grund von Zitationszahlen/etc) hochpuschen wollen


% Kozitation:
% Def: 2 Publikationen werden gemeinsam in einer dritten zitiert


lalala

\subsection{Semantische Netzwerke}
% Was sind semantische Netzwerke
% Warum sind sie in unserem Fall wichtig?/Warum haben wir uns für so eine Wissensrepräsentation entschieden?

\subsection{Verwandte Arbeiten}
% Was gibt es schon zum Thema semantische Modellierung von Publikationen (Science Ontology, Diplomarbeit vom einen Typen)
% Was gibt es schon zum Thema Ähnlichkeitsmaßen (Kozitationen, Koautorenschaften, bibliographische Kopplung, Ähnlichkeit zwischen Autoren: \cite{Yang:2009:TRSN})

