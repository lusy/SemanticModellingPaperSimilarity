\section{Zusammenfassung}

\subsection{Ausblick}


%* was haben wir gemacht
%* was gewinnen wir durch semantische Analyse von Metadaten
%* hinweis auf weiterf�hrende recherche

% Koennen wir auch weitere Daten analysieren? Auf anderen Weisen (nicht semantisch)?
% zb mathematische formeln/theoreme/diagramme, die im text vorkommen
% Wie kann man das �hnlichkeitsma� anders/besser definieren?

% man kann das semantische �hnlichkeitsma� mit einem textbasierten kombinieren, das die �hnlichkeit zwischen Title/Abstracts auswertet, und diese dann auch f�r die Semantische �hnlichkeit nutzen
% Levenshtein distance: einfache Distanz zwischen 2 Strings/Texten \cite ???
% basiert auf die Anzahl von Einf�gungen, L�schungen, Ersetzungen von Buchstaben, die gebraucht werden, um einen String in einen anderen zu transformieren
% extrem unpraktisch: wir k�nnen nicht f�r tausende von Dokumenten, mit mehreren Seiten Text die Levenshteindistanz messen..
% fragw�rdig, ob brauchbare Ergebnisse rauskommen (wobei man k�nnte vlt so was schon f�r das Bauen einer Plagiatsoftware einsetzen..)
% kann vlt f�r das vergleichen von Abstracts/Titel genutzt werden


% weiterf�hrend kann man die Performance/die konkrete Implementierung, die die Performance ausmacht verbessern - daf�r muss man die Performance evaluieren!
% Theoretische Grenzen: Speicher, Zeit
% wie siehts praktisch aus? symmetrische Sparsematrix (Python dictionary) mit nur den Nichtnull-Eintr�gen wird gespeichert
% Laufzeit: die �hnlichkeit f�r verschiedenartige Knoten wird nicht weiter verfolgt, da wird auch was geprunnt.
%%----------- Eigentlich sind die Sachen unten f�r den Ausblick!
% wenn G zu gro�, kann mans extern, au�erhalb des Hauptspeichers auslagern (z.b. Graphdatenbank) (die �hnlichkeitsmatrix, egal welche Struktur f�r ihre Repr�sentation gew�hlt wird, wird erstmal einfachshalber im Arbeitsspeicher behalten
% das eigentliche (gr��ere) Problem ist die Laufzeit -- es k�nnen Verfahren gesucht werden, um die zu optimieren / Teile des Algorithmus zu parallelisieren


% man kann doch als weiterf�hrende Forschung eben solche Anwendungen bauen, die auf so einem �hnlichkeitsma� beruhen
% man kann das Ma� auch daf�r nutzen, zeitliche Analyse der Entwicklung der Wissenschaft in bestimmten Forschungsfelder/zu bestimmten Forschungsthemen zu entwerfen




%In a general way,
%
%restate your topic and why it is important,
%restate your thesis/claim,
%address opposing viewpoints and explain why readers should align with your position,
%call for action or overview future research possibilities.
%
%from my specific topic to more general matters
%say what you have already said but do it quickly, sharply and in different words
