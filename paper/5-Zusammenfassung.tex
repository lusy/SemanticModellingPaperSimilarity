\section{Zusammenfassung}

\subsection{Ausblick}


%* was haben wir gemacht
%* was gewinnen wir durch semantische Analyse von Metadaten
%* hinweis auf weiterf�hrende recherche

% Koennen wir auch weitere Daten analysieren? Auf anderen Weisen (nicht semantisch)?
% zb mathematische formeln/theoreme/diagramme, die im text vorkommen
% Wie kann man das �hnlichkeitsma� anders/besser definieren?

% man kann das semantische �hnlichkeitsma� mit einem textbasierten kombinieren, das die �hnlichkeit zwischen Title/Abstracts auswertet, und diese dann auch f�r die Semantische �hnlichkeit nutzen
% weiterf�hrend kann man die Performance/die konkrete Implementierung, die die Performance ausmacht verbessern - daf�r muss man die Performance evaluieren!

% man kann doch als weiterf�hrende Forschung eben solche Anwendungen bauen, die auf so einem �hnlichkeitsma� beruhen
% man kann das Ma� auch daf�r nutzen, zeitliche Analyse der Entwicklung der Wissenschaft in bestimmten Forschungsfelder/zu bestimmten Forschungsthemen zu entwerfen




%In a general way,
%
%restate your topic and why it is important,
%restate your thesis/claim,
%address opposing viewpoints and explain why readers should align with your position,
%call for action or overview future research possibilities.
%
%from my specific topic to more general matters
%say what you have already said but do it quickly, sharply and in different words
