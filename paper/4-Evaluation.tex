\section{Evaluation}

\subsection{Testl�ufe}
% Anwendung des �hnlichkeitsma�es
% Was hats gebracht/wie siehts aus?

% Parameters of the test system:
%Hardware Overview:
%    Model Name: Mac Pro
%    Model Identifier: MacPro4,1
%    Processor Name: Quad-Core Intel Xeon
%    Processor Speed: 2,93 GHz
%    Number Of Processors: 2
%    Total Number Of Cores: 8
%    L2 Cache (per core): 256 KB
%    L3 Cache (per processor): 8 MB
%    Memory: 14 GB
%    Processor Interconnect Speed: 6.4 GT/s

%System Software Overview:
%    System Version: Mac OS X 10.6.8 (10K549)
%    Kernel Version: Darwin 10.8.0
%    Boot Volume: Macintosh HD
%    Boot Mode: Normal



\subsection{Auswertung der Ergebnisse}
% Macht das, was rauskommt, Sinn?

% Vergleich gegen die MSC-Klassen
% Vergleich mit einem rein bibliometrischen Verfahren (bibliographische Kopplung / SimRank/ ..)
% Entwickle 3 Varianten und vergleich sie: mit unterschiedlichen Gewichtung von den verschiedenen Relationen
% Idee dahinter: wenn alle Relationen gleich gewichtet sind, dann k�nnen wir genau so gut auch keinen Unterschied zwischen denen machen
% Aber wir k�nnen die gleichgewichtete Variante als eine von den 3 nehmen

% evaluiere Performance (Zeit, Platz, ..)

% vgl mit den anderen Ma�en (SimRank, P-Rank, C-Rank): auf wie vielen Daten wurden diese angewandt? Wie viele Ressourcen hat die Berechnung in Anspruch genommen?

% Datensatz trimmen: weil so lange
% Klar machen nach welchem Prinzip getrimmt wird und was die Nachteile davon sind (Relationen gehen verloren)
% Andererseits wenn nur fr�here Publikationen betrachtet werden, werden zumindest keine Zitationsrelatioonen zu sp�teren Papers verloren
% Trimmen ist trotzdem schei�e, weil sich Zitationsverhalten so wie die Tendenzen zu gr��eren Koautorenschaften  mit der Zeit ver�ndert
% klar machen dass der Algorithmus in Zeit O(n^4) l�uft und warum (Pseudocode)
