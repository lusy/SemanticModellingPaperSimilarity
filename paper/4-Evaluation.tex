\section{Evaluation}

\subsection{Komplexit�tsanalyse des Algorithmus}
% Pseudo code
% speicher- und laufzeitkomplexit�t kl�ren

\subsection{Testl�ufe}
% Vergleich: Laufzeit und Speicherverbrauch der alten strukturellen Ma�e (SimRank, P-Rank, C-Rank)
% auf wie vielen Daten wurden diese ausgewertet und wie lange hats gedauert? was f�r parameter hatte das testsystem?

% Was genau wurde ausgerechnet?

%Um das vorgeschlagene �hnlichkeitsma� f�r wissenschaftliche Publikationen auswerten zu k�nnen, wird der Testdatensatz vom Zentralblatt Mathematik auf das im Kapitel \ref{subsec:modell} vorgestellte Graphenschema abgebildet.
%Mit Hilfe von einem in \textit{Python} geschriebenen Parser werden die Daten vom \textit{zmath}-Datensatz in \textit{GraphML}-Format �berf�hrt.
%Die \textit{GraphML}-Repr�sentation der vollst�ndigen Daten ist ca. $4.8$ Gb gro�.

% 1. Trimming des Datensatzes: auf $1$ Gb - nur die Publikationen mit vollst�ndigen Metadaten, $1,154,950$ Publikationen
% -- nicht wirklich berechenbar in vern�nftiger Zeit (siehe Komplexit�tsanalyse)
% 2. Trimming des Datensatzes: 3095 Publikationen mit vollst�ndigen Metadaten (alle Publikationen im Datensatz, die vor dem Jahr 1975 ver�ffentlicht wurden);
% So werden nicht Zitationsrelationen zu sp�teren Arbeiten verloren (eine fr�here Arbeit kann eine sp�tere nicht zitieren)
% andererseit aber werden Relationen verloren, da der Algorithmus auf dem Gesamtpublikationsnetzwerk rechnet (also Werte, die �ber rausgeschnittenen Knoten propagiert wurden, gehen verloren)
% -- also eine Auswertung "wie gut ist das �hnlichkeitsma�" wird vermutlich nicht sehr sinnvolle Ergebnisse liefern


% Parameters of the test system:
%Hardware Overview:
%    Model Name: Mac Pro
%    Model Identifier: MacPro4,1
%    Processor Name: Quad-Core Intel Xeon
%    Processor Speed: 2,93 GHz
%    Number Of Processors: 2
%    Total Number Of Cores: 8
%    L2 Cache (per core): 256 KB
%    L3 Cache (per processor): 8 MB
%    Memory: 14 GB
%    Processor Interconnect Speed: 6.4 GT/s

%System Software Overview:
%    System Version: Mac OS X 10.6.8 (10K549)
%    Kernel Version: Darwin 10.8.0
%    Boot Volume: Macintosh HD
%    Boot Mode: Normal

\subsection{Bestimmung der Parameter}
% wie werden die lambdas gew�hlt? warum?
% wie wird c gew�hlt? warum (simRank optimierungspaper)

% Notes on accuracy, decay factor and convergence:
% ------------------------------------------------

% Accuracy: C-Rank > SimRank > P-Rank > rvs-SimRank

% Number of iterations:
% SimRank converges at k = 3,
% rvs-SimRank converges at k = 5,
% P-Rank converges at k = 6,
% C-Rank converges at k = 9.

% Decay factor:
% It is obvious that the similarity score of C-Rank increases with the increase of C.
% When C = 0.2, C-Rank converges fast at k = 2.
% When C = 0.8, on the other hand, C-Rank converges at the 9-th iteration.
% When C is low, the recursive power of C-Rank is weakened such that only the papers in local or near-local neighborhood are used in similarity computation.
% When C is high, more papers in a more global neighborhood can be used in computing the similarity recursively. When C is high, therefore, the convergence takes more time.


\subsection{Auswertung der Ergebnisse}
% Macht das, was rauskommt, Sinn?

% Vergleich gegen die MSC-Klassen
% Vergleich mit einem rein bibliometrischen Verfahren (bibliographische Kopplung / SimRank/ ..)
% Entwickle 3 Varianten und vergleich sie: mit unterschiedlichen Gewichtung von den verschiedenen Relationen

% Clustering �ber die entstandene �hnlichkeitsmatrix f�r alle Verfahren (C-Rank + alle 3 Parametrisierungen)
% Idee: Vergleich entstandene Cluster mit den urspr�nglich vergebenen MSC-Klassen: wenn die MSC-Klassifizierung gut abgebildet, gutes Clustering
% MSC Klassen werden bis zur Top-Level Klassen aggregiert
% Wahl des Clusteringverfahrens

% Ergebnisse: Durchschnittswerte von Entropy, Purity, Silhouette-Koeffizient, Verteilung
%% Kurze Definition von Entropy, Purity und Silhouette
%% Beschreibung/Vergleich der Ergebnisse (f�r 3095 Publikationen und 64 Cluster)
% -- SemSim schneidet schon besser als C-Rank ab (C-Rank packt alles in den selben Cluster)
% -- Schlussfolgerungen: Entweder war das Clusteringverfahren doof oder aber ist das Trimming vom Datensatz dumm und es k�nnen keine ad�quate Ergebnisse geliefert werden
