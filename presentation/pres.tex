\documentclass[12pt, xcolor=table]{beamer}
\usepackage{graphicx}
\usepackage[ngerman]{babel}
\usepackage[utf8]{inputenc}
\usepackage{amsmath}
\usepackage{amssymb}
\usepackage{listings}
\usepackage{hyperref}
\usepackage{fancyvrb}
\usepackage{color}

\usepackage[percent]{overpic}
\usepackage[footnotesize, bf]{caption}
\input{theme.tex}
\input{syntax}
\renewcommand{\footnotesize}{\tiny}


\begin{document}
\title{Semantische Modellierung von mathematischen Publikationen und Definition eines Ähnlichkeitsmaßes}
\author[Lyudmila Vaseva]{Lyudmila Vaseva\\{vaseva@mi.fu-berlin.de}}
%\email{vaseva@mi.fu-berlin.de}
\institute{Freie Universität Berlin \\
    Institut für Informatik}
\date{\today}

\begin{frame}
	\titlepage
\end{frame}

%Motivation! (ohne Folie)
\begin{frame}
    \begin{block}{Motivation}
%    \begin{itemize}
%
%    \end{itemize}
    \end{block}
    \begin{block}{Ziel}
    \begin{itemize}
    \item Präzisere Ähnlichkeitsberechnung für Publikationen aufgrund möglichst vieler Metadaten
    \end{itemize}
    \end{block}
\end{frame}

\begin{frame}
    \frametitle{Gliederung}
    \begin{enumerate}
        \item Existierende Repräsentationen
        \item Algorithmen zur Bestimmung der Ähnlichkeit von wissenschaftlichen Arbeiten
        \item Publikationsnetzwerk
        \item Semantische Ähnlichkeit
        \item Laufzeitanalyse
        \item Ergebnisse und Auswertung
    \end{enumerate}
\end{frame}

%Vlt weglassen
%\begin{frame}
%    \frametitle{Existierende Repräsentationen von Publikationen}
%\end{frame}

\begin{frame}
    \frametitle{Existierende Ähnlichkeitsmaße - textbasiert}
    Zum Beispiel das auf TF-IDF basierte Kosinusmaß
    \begin{block}{Nachteile}
    \begin{itemize}
        \item sprachabhängig
        \item nicht praktikabel für große Dokumente/Dokumentenmengen
        \item Semantik der Terme wird nicht erfasst
    \end{itemize}
    \end{block}
    %Vor- und Nachteile
\end{frame}

%Gleichungen?/Tabelle mit Überblick?
\begin{frame}
    \frametitle{Existierende Ähnlichkeitsmaße - zitationsbaseirt I}
\end{frame}

\begin{frame}
    \frametitle{Existierende Ähnlichkeitsmaße - zitationsbasiert II}
    \begin{block}{Weitere Beispiele}
    \begin{itemize}
        \item CC-IDF
        \item Lu et al.: Ähnlichkeit aufgrund lokaler Nachbarschaft
    \end{itemize}
    \end{block}
\end{frame}

\begin{frame}
    \frametitle{Existierende Ähnlichkeitsmaße - zitationsbasiert III}
    %Vor- und Nachteile
    \begin{block}{Nachteile}
    \begin{itemize}
        \item Gründe für das Zitieren unbekannt
        %spezifische Nachteile der einzelnen Ansätze
    \end{itemize}
    \end{block}
\end{frame}

\begin{frame}
    \frametitle{Publikationsnetzwerk}
    %Bild!
    \begin{center}
        \includegraphics[scale=0.38]{../deps/publications_ontology_relations.png}
    \end{center}
\end{frame}


\begin{frame}
    \frametitle{Semantische Ähnlichkeit}
    Formel!
\end{frame}

\begin{frame}
    \frametitle{Evaluation}
    \begin{itemize}
        \item Laufzeit: $\mathcal{O}(n^4)$
        \item Speicherkomplexität: $\mathcal{O}(n^2)$
        \item Der Testdatensatz enthält ursprünglich $2,907,086$ Publikationen
        \item Der Testdatensatz wird auf $3095$ Publikationen, $15,994$ Knoten insgesamt gekürzt
    \end{itemize}
\end{frame}

\begin{frame}
    \frametitle{Ergebnisse des semantischen Ähnlichkeitsmaßes}
    \begin{itemize}
        \item Vergleich mit \textit{C-Rank}
    \end{itemize}
\end{frame}

\begin{frame}
    \frametitle{Fazit}
    \begin{itemize}
    \item Die semantische Ähnlichkeit liefert bessere Ergebnisse als Ansätze, die nur auf einer Art Metadaten basieren
    \item Semantische Ähnlichkeit sollte für das vollständige Publikationsnetzwerk berechnet werden
    \item Der Ansatz ist leider nicht praktikabel für große Datenmengen
        \begin{itemize}
            \item Laufzeitoptimierung durch Parallelisierung?
            \item Laufzeitoptimierung durch nicht-iterative Berechnung?
            \item Speicheroptimierung durch Auslagern der Daten in eine Datenbank
        \end{itemize}
    \end{itemize}
\end{frame}

\begin{frame}
\centerline{Vielen Dank!}
\end{frame}

%\begin{frame}
%	\frametitle{Eigenschaften des Datensatzes}
%	\begin{itemize}
%		\item  enthält ca. $706\,000$ Einträge
%		\item  mit 19 verschiedenen Themengebieten
%		\item  nur der Themenbereich Physik wird in Themengruppen unterteilt
%		\item  11 Einträge ohne Informationen
%		\item  Publikationen haben im Durchschnitt 1.3 und maximal 9 Themen
%	\end{itemize}
%\end{frame}

\end{document}
