\documentclass[12pt, xcolor=table]{beamer}
\usepackage{graphicx}
\usepackage[ngerman]{babel}
\usepackage[utf8]{inputenc}
\usepackage{amsmath}
\usepackage{amssymb}
\usepackage{listings}
\usepackage{hyperref}
\usepackage{fancyvrb}
\usepackage{color}
\usepackage{comment}

\usepackage[percent]{overpic}
\usepackage[footnotesize, bf]{caption}
\input{theme.tex}
\input{syntax}
\renewcommand{\footnotesize}{\tiny}


\begin{document}
\title{Semantische Modellierung von mathematischen Publikationen und Definition eines Ähnlichkeitsmaßes}
\author[Lyudmila Vaseva]{Lyudmila Vaseva\\{vaseva@mi.fu-berlin.de}}
%\email{vaseva@mi.fu-berlin.de}
\institute{Freie Universität Berlin \\
    Institut für Informatik}
\date{\today}

\begin{frame}
	\titlepage
\end{frame}


\begin{frame}
    \frametitle{Gliederung}
    \begin{enumerate}
        \item Motivation
        \item Existierende Algorithmen zur Bestimmung der Ähnlichkeit von wissenschaftlichen Arbeiten
        \item Publikationsnetzwerk
        \item Semantische Ähnlichkeit
        \item Semantische Ähnlichkeit: Analyse, Ergebnisse und Auswertung
        \item Fazit
    \end{enumerate}
\end{frame}

%Vlt weglassen
%\begin{frame}
%    \frametitle{Existierende Repräsentationen von Publikationen}
%\end{frame}

%Motivation! (ohne Folie)
\begin{frame}
    \begin{block}{Motivation}
    \begin{itemize}
        \item Recommender Systeme
        \item Automatisierte Klassifikation
    \end{itemize}
    \end{block}

    \begin{block}{Ziel}
    \begin{itemize}
    \item Präzisere Ähnlichkeitsberechnung für Publikationen aufgrund möglichst vieler Metadaten
    \end{itemize}
    \end{block}
\end{frame}

\begin{frame}
    \frametitle{Existierende Ähnlichkeitsmaße}
    \begin{columns}[T]
        \begin{column}{0.45\textwidth}
        \begin{block}{textbasiert}
            \begin{itemize}
                \item sprachabhängig
                \item unbekannte Semantik der Terme
                \item nicht praktikabel bei großen Datensätzen
            \end{itemize}
        \end{block}
        \end{column}

        \begin{column}{0.45\textwidth}
        \begin{block}{zitationsbasiert}
            \begin{itemize}
                \item Grund für Zitieren unbekannt
            \end{itemize}
        \end{block}
        \end{column}
    \end{columns}

%    \begin{table}[h]
%   \centering
%    \begin{tabular}{p{6cm} p{6cm}}
%    textbasiert & zitationsbasiert\\
%    \hline
%    sprachabhängig & Grund für Zitieren unbekannt\\
%    unbekannte Semantik der Terme & \\
%    nicht praktikabel bei großen Datensätzen\\
%\end{tabular}
%\end{table}
\end{frame}

\begin{comment}
\begin{frame}
    \frametitle{Existierende Ähnlichkeitsmaße - textbasiert}
    Beispiel: Kosinusmaß basierend auf TF-IDF
    \begin{block}{Nachteile}
    \begin{itemize}
        \item sprachabhängig
        \item nicht praktikabel für große Dokumente/Dokumentenmengen
        \item Semantik der Terme wird nicht erfasst
    \end{itemize}
    \end{block}
    %Vor- und Nachteile
\end{frame}


%Gleichungen?/Tabelle mit Überblick?
\begin{frame}
    \frametitle{Existierende Ähnlichkeitsmaße - zitationsbaseirt I}
    \begin{block}{\small{Bibiliographische Kopplung}}
    \small{
        $sim(a,b) = O(a) \cap O(b)$
    }
    \end{block}

    \begin{block}{\small{Kozitation}}
    \small{
         $sim(a,b) = I(a) \cap I(b)$
    }
    \end{block}

    \begin{block}{\small{Amsler}}
    \small{
        $sim(a,b) = \cfrac{L(a) \cap L(b)}{L(a) \cup L(b)}$
    }
    \end{block}
\end{frame}

\begin{frame}
    \frametitle{Existierende Ähnlichkeitsmaße - zitationsbasiert II}
    \begin{block}{\small{simRank}}
    \scriptsize{
        $sim(a,b) = 1 \quad \text{wenn } a=b$
        \newline
        sonst: \quad
        $sim(a,b)  = 
        \cfrac{c}{|I(a)||I(b)|}
        \sum_{i=1}^{|I(a)|} \sum_{j=1}^{|I(b)|} sim(I_i(a),I_j(b))$
    }
    \end{block}

    \begin{block}{\small{P-Rank}}
    \scriptsize{
        $sim(a,b) = 1 \quad \text{wenn } a=b$
        \newline
        sonst: \quad
        $sim(a,b)  = 
        \lambda\times\cfrac{c}{|I(a)||I(b)|}
        \sum_{i=1}^{|I(a)|} \sum_{j=1}^{|I(b)|} sim(I_i(a),I_j(b))
         +
        (1-\lambda)\times\cfrac{c}{|O(a)||O(b)|}
        \sum_{i=1}^{|O(a)|} \sum_{j=1}^{|O(b)|} sim(O_i(a),O_j(b))$
    }
    \end{block}

    \begin{block}{\small{C-Rank}}
    \scriptsize{
        $sim(a,b) = 1 \quad \text{wenn } a=b$
        \newline
        sonst: \quad
        $sim(a,b)  = 
        \cfrac{c}{|L(a)||L(b)|}
        \sum_{i=1}^{|L(a)|} \sum_{j=1}^{|L(b)|} sim(L_i(a),L_j(b))$
   }
    \end{block}

\end{frame}


\begin{frame}
    \frametitle{Existierende Ähnlichkeitsmaße - zitationsbasiert III}
    \begin{block}{Weitere Beispiele}
    \begin{itemize}
        \item CC-IDF
        \item Lu et al.: Ähnlichkeit aufgrund lokaler Nachbarschaft
    \end{itemize}
    \end{block}
\end{frame}

\begin{frame}
    \frametitle{Existierende Ähnlichkeitsmaße - zitationsbasiert IV}
    %Vor- und Nachteile
    \begin{block}{Nachteile}
    \begin{itemize}
        \item Gründe für das Zitieren unbekannt
        %spezifische Nachteile der einzelnen Ansätze
    \end{itemize}
    \end{block}
\end{frame}
\end{comment}

\begin{frame}
    \frametitle{Publikationsnetzwerk}
    %Bild!
    \begin{center}
        \includegraphics[scale=0.38]{../deps/publications_ontology_relations.png}
    \end{center}
\end{frame}


\begin{frame}
    \frametitle{Semantische Ähnlichkeit}
    %Formel!
    %Global auf das Netzwerk
    %Rekursive Natur
    %kann/wird iterativ berechnet, konvergiert zum Fixpunkt
    %Für jedes Elementenpaar: schaue alle Elementenpaare, die sich aus den Nachbarn bilden an -- Laufzeit
    %Verschiedenartige Metadaten werden durch Parameter gewichtet
    %Dämpfungskonstante, die bestimmt wie stark weiter entfernte Knoten die aktuellen Knoten beeinflussen
\scriptsize{
\[
 R_0(a,b) =
    \begin{cases}
     1 & \text{wenn } $a$ = $b$ \\
     0 & \text{sonst}\\
    \end{cases}
\]
\newline
\text{wenn a und b Publikationen:}
\[
\begin{array}{lcl}
 R_{k+1}(a,b) & = & 
        \lambda_1\times\cfrac{c}{|K(a)||K(b)|}
        \sum_{i=1}^{|K(a)|} \sum_{j=1}^{|K(b)|} R_k(K_i(a),K_j(b))
        \\ & + &
        \lambda_2\times\cfrac{c}{|A(a)||A(b)|}
        \sum_{i=1}^{|A(a)|} \sum_{j=1}^{|A(b)|} R_k(A_i(a),A_j(b))
        \\ & + &
        \lambda_3\times\cfrac{c}{|S(a)||S(b)|}
        \sum_{i=1}^{|S(a)|} \sum_{j=1}^{|S(b)|} R_k(S_i(a),S_j(b))
        \\ & + &
        \lambda_4\times\cfrac{c}{|C(a)||C(b)|}
        \sum_{i=1}^{|C(a)|} \sum_{j=1}^{|C(b)|} R_k(C_i(a),C_j(b))
        \\ & + &
        \lambda_5\times\cfrac{c}{|Y(a)||Y(b)|}
        \sum_{i=1}^{|Y(a)|} \sum_{j=1}^{|Y(b)|} R_k(Y_i(a),Y_j(b))
\end{array}
\]
\newline
\text{wenn a und b aus sonstiger Klasse:}
\[
R_{k+1}(a,b)  = 
        \cfrac{c}{|L(a)||L(b)|}
        \sum_{i=1}^{|L(a)|} \sum_{j=1}^{|L(b)|} R_k(L_i(a),L_j(b))
\]
}
\end{frame}

\begin{frame}
    \frametitle{Evaluation}
    \begin{itemize}
        \item Laufzeit: $\mathcal{O}(n^4)$
        \item Speicherkomplexität: $\mathcal{O}(n^2)$
        \item Der Testdatensatz enthält ursprünglich $2,907,086$ Publikationen
        \item Der Testdatensatz wird auf $3095$ Publikationen, $15,994$ Knoten insgesamt gekürzt - alle Publikationen bis 1975
    \end{itemize}
\end{frame}

\begin{frame}
    \frametitle{Ergebnisse des semantischen Ähnlichkeitsmaßes}
    \scriptsize{
    \begin{table}[H]
    \centering
    \begin{tabular}{l c c c}
		%\hline
		\textbf{Parameter} &\textbf{Gewichtung 1} &\textbf{Gewichtung 2} &\textbf{Gewichtung 3} \\
		%\hline
		$\lambda_1$, Keywords & $0.5$ & $0.2$ & $0.4$\\
		$\lambda_2$, Autoren & $0.1$ & $0.2$ & $0.3$\\
		$\lambda_3$, Quellen & $0.1$ & $0.2$ & $0.1$\\
		$\lambda_4$, Zitationen & $0.2$ & $0.2$ & $0.1$\\
		$\lambda_5$, Erscheinungsjahre & $0.1$ & $0.2$ & $0.1$\\
	    %\hline
    \end{tabular}
    \end{table}
    }
    \scriptsize{
    \begin{table}[H]
    \centering
    \begin{tabular}{l l l l l}
	    	%\hline
		     &\textbf{Gewichtung 1} &\textbf{Gewichtung 2} &\textbf{Gewichtung 3} & \textbf{C-Rank} \\
    		%\hline
	    	Entropy & $2.4863$ & $3.28742$ & $2.48117$ & $4.13098$\\
		    Purity & $0.50953$ & $0.39774$ & $0.51599$ & $0.35767$\\
    		Silhouette & $0.084$ & $0.056$ & $0.069$ & $0.0$ \\
	        %\hline
    \end{tabular}
\end{table}
}
%   \begin{itemize}
%        \item Vergleich mit \textit{C-Rank}
%    \end{itemize}
\end{frame}

\begin{frame}
    \frametitle{Fazit}
    \begin{itemize}
    \item bessere Ergebnisse als Maße, die nur auf einer Art Metadaten basieren
    \item Datensatzverkürzung beeinträchtigt die Ergebnisse
    \item nicht praktikabel für große Datenmengen
        \begin{itemize}
            \item Laufzeitoptimierung durch Parallelisierung?
            \item Laufzeitoptimierung durch nicht-iterative Berechnung?
            \item Speicheroptimierung durch Auslagern der Daten in eine Datenbank
        \end{itemize}
    \end{itemize}
\end{frame}

\begin{frame}
\centerline{Vielen Dank!}
\end{frame}

\end{document}
