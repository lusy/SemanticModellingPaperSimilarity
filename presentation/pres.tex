\documentclass[12pt, xcolor=table]{beamer}
\usepackage{graphicx}
\usepackage[ngerman]{babel}
\usepackage[utf8]{inputenc}
\usepackage{amsmath}
\usepackage{amssymb}
\usepackage{listings}
\usepackage{hyperref}
\usepackage{fancyvrb}
\usepackage{color}

\usepackage[percent]{overpic}
\usepackage[footnotesize, bf]{caption}
%Copyright 2008 by Adrian Böhmichen
%
% This file is free software: you can redistribute it and/or modify
% it under the terms of the GNU General Public License as published by
% the Free Software Foundation, either version 3 of the License, or
% (at your option) any later version.
%
% This file is distributed in the hope that it will be useful,
% but WITHOUT ANY WARRANTY; without even the implied warranty of
% MERCHANTABILITY or FITNESS FOR A PARTICULAR PURPOSE.  See the
% GNU General Public License for more details.
%
% You should have received a copy of the GNU General Public License
% along with this file.  If not, see <http://www.gnu.org/licenses/>.

%%%%%%%%%%%%%%%%%%%%%%%%%%%%%%%%%%%%%%%%%%%%%%%%%%%%%%%%%%%%%%%%%
%     Ubuntuusers Vorlage für ein LaTeX-Beamer Theme            %
%                                                               %
% Für das Korrekte funktionieren benötigt man einen header.png  %
% und ein logo.png Datei!                                       %
% Zusätzlich muss man folgende Pakete benutzten:                %
%   \usepackage{graphicx}                                       %
%   \usepackage[percent]{overpic}                               %
%                                                               %
% Danach muss nur noch am Anfang die Datei                      %
% mit \input{} eingebunden werden.                              %
%                                                               %
%%%%%%%%%%%%%%%%%%%%%%%%%%%%%%%%%%%%%%%%%%%%%%%%%%%%%%%%%%%%%%%%%

%weitere Farbe spezifizieren:
%Farben von dem Humantheme
%\definecolor{Orange}{RGB}{240,165,19}
\definecolor{Orange}{RGB}{5,215,242}
%\definecolor{Human-Base}{RGB}{129,102,71}
\definecolor{Human-Base}{RGB}{5,25,242}
%Farben aus dem Inyokatheme
%\definecolor{uuheader1}{RGB}{164,143,101}
\definecolor{uuheader1}{RGB}{5,25,242}
%\definecolor{uuheader2}{RGB}{129,106,59}
\definecolor{uuheader2}{RGB}{5,25,242}


%Theme festlegen für alle Templates die nicht selbstständig definiert werden:
\usepackage{beamerthemedefault}


%Definieren des Innertheme, zuständig für die Symbole bei Listen
\setbeamertemplate{sections/subsections in toc}[square]
\setbeamertemplate{items}[circle]

\setbeamercolor{item}{fg=Human-Base}

%entfernen der Navigationsleiste
\beamertemplatenavigationsymbolsempty

%Logo definieren, man kann die Lage nicht verändern
%\logo{\includegraphics[scale=0.1]{logo.png}}


%Kopf- und Fußzeile definieren
%\setbeamertemplate{headline}
%{%
%\begin{overpic}[width=\paperwidth
% nächste Zeile dient zum anzeigen eines Rasters, für das paltzieren des ToC hilfreich
%,grid,tics=10
%]
%{header.png}%
%  \put(0,11){\insertsectionnavigationhorizontal{\paperwidth}{~}{~}}%
%  \end{overpic}
%}

\setbeamertemplate{footline}[text line]
{%
\begin{minipage}[b]{116mm}
\insertauthor \hfill%
%neue Navigationsleiste
 \insertframenumber ~/ \inserttotalframenumber\\[1ex]
\end{minipage}
}

% Farben festlegen ausserhalb des innertheme

%Allgemeine Angaben und Verbesserung vom default Theme
\setbeamercolor{structure}{fg=uuheader1}
\setbeamercolor{section in toc}{fg=Human-Base}
\setbeamercolor{subsection in toc}{parent=section in toc}
\setbeamercolor{framesubtitle}{fg=uuheader2}


%Farbe und Form der Blöcke definieren
\setbeamertemplate{blocks}[rounded]
%\setbeamercolor{block title}{fg=uuheader1,bg=Orange}
%\setbeamercolor{block title alerted}{use=alerted text,fg=black,bg=alerted text.fg!75!bg}
%\setbeamercolor{block title example}{use=example text,fg=black,bg=example text.fg!75!bg}

%\setbeamercolor{block body}{parent=normal text,use=block title,bg=block title.bg!25!bg}
%\setbeamercolor{block body alerted}{parent=normal text,use=block title alerted,bg=block title alerted.bg!25!bg}
%\setbeamercolor{block body example}{parent=normal text,use=block title example,bg=block title example.bg!25!bg}

%Für den Titleframe
\setbeamertemplate{title page}[default][rounded=true]
\setbeamercolor{title}{fg=uuheader2,bg=Orange}


\makeatletter
\def\PY@reset{\let\PY@it=\relax \let\PY@bf=\relax%
    \let\PY@ul=\relax \let\PY@tc=\relax%
    \let\PY@bc=\relax \let\PY@ff=\relax}
\def\PY@tok#1{\csname PY@tok@#1\endcsname}
\def\PY@toks#1+{\ifx\relax#1\empty\else%
    \PY@tok{#1}\expandafter\PY@toks\fi}
\def\PY@do#1{\PY@bc{\PY@tc{\PY@ul{%
    \PY@it{\PY@bf{\PY@ff{#1}}}}}}}
\def\PY#1#2{\PY@reset\PY@toks#1+\relax+\PY@do{#2}}

\def\PY@tok@gd{\def\PY@tc##1{\textcolor[rgb]{0.63,0.00,0.00}{##1}}}
\def\PY@tok@gu{\let\PY@bf=\textbf\def\PY@tc##1{\textcolor[rgb]{0.50,0.00,0.50}{##1}}}
\def\PY@tok@gt{\def\PY@tc##1{\textcolor[rgb]{0.00,0.25,0.82}{##1}}}
\def\PY@tok@gs{\let\PY@bf=\textbf}
\def\PY@tok@gr{\def\PY@tc##1{\textcolor[rgb]{1.00,0.00,0.00}{##1}}}
\def\PY@tok@cm{\let\PY@it=\textit\def\PY@tc##1{\textcolor[rgb]{0.25,0.50,0.50}{##1}}}
\def\PY@tok@vg{\def\PY@tc##1{\textcolor[rgb]{0.10,0.09,0.49}{##1}}}
\def\PY@tok@m{\def\PY@tc##1{\textcolor[rgb]{0.40,0.40,0.40}{##1}}}
\def\PY@tok@mh{\def\PY@tc##1{\textcolor[rgb]{0.40,0.40,0.40}{##1}}}
\def\PY@tok@go{\def\PY@tc##1{\textcolor[rgb]{0.50,0.50,0.50}{##1}}}
\def\PY@tok@ge{\let\PY@it=\textit}
\def\PY@tok@vc{\def\PY@tc##1{\textcolor[rgb]{0.10,0.09,0.49}{##1}}}
\def\PY@tok@il{\def\PY@tc##1{\textcolor[rgb]{0.40,0.40,0.40}{##1}}}
\def\PY@tok@cs{\let\PY@it=\textit\def\PY@tc##1{\textcolor[rgb]{0.25,0.50,0.50}{##1}}}
\def\PY@tok@cp{\def\PY@tc##1{\textcolor[rgb]{0.74,0.48,0.00}{##1}}}
\def\PY@tok@gi{\def\PY@tc##1{\textcolor[rgb]{0.00,0.63,0.00}{##1}}}
\def\PY@tok@gh{\let\PY@bf=\textbf\def\PY@tc##1{\textcolor[rgb]{0.00,0.00,0.50}{##1}}}
\def\PY@tok@ni{\let\PY@bf=\textbf\def\PY@tc##1{\textcolor[rgb]{0.60,0.60,0.60}{##1}}}
\def\PY@tok@nl{\def\PY@tc##1{\textcolor[rgb]{0.63,0.63,0.00}{##1}}}
\def\PY@tok@nn{\let\PY@bf=\textbf\def\PY@tc##1{\textcolor[rgb]{0.00,0.00,1.00}{##1}}}
\def\PY@tok@no{\def\PY@tc##1{\textcolor[rgb]{0.53,0.00,0.00}{##1}}}
\def\PY@tok@na{\def\PY@tc##1{\textcolor[rgb]{0.49,0.56,0.16}{##1}}}
\def\PY@tok@nb{\def\PY@tc##1{\textcolor[rgb]{0.00,0.50,0.00}{##1}}}
\def\PY@tok@nc{\let\PY@bf=\textbf\def\PY@tc##1{\textcolor[rgb]{0.00,0.00,1.00}{##1}}}
\def\PY@tok@nd{\def\PY@tc##1{\textcolor[rgb]{0.67,0.13,1.00}{##1}}}
\def\PY@tok@ne{\let\PY@bf=\textbf\def\PY@tc##1{\textcolor[rgb]{0.82,0.25,0.23}{##1}}}
\def\PY@tok@nf{\def\PY@tc##1{\textcolor[rgb]{0.00,0.00,1.00}{##1}}}
\def\PY@tok@si{\let\PY@bf=\textbf\def\PY@tc##1{\textcolor[rgb]{0.73,0.40,0.53}{##1}}}
\def\PY@tok@s2{\def\PY@tc##1{\textcolor[rgb]{0.73,0.13,0.13}{##1}}}
\def\PY@tok@vi{\def\PY@tc##1{\textcolor[rgb]{0.10,0.09,0.49}{##1}}}
\def\PY@tok@nt{\let\PY@bf=\textbf\def\PY@tc##1{\textcolor[rgb]{0.00,0.50,0.00}{##1}}}
\def\PY@tok@nv{\def\PY@tc##1{\textcolor[rgb]{0.10,0.09,0.49}{##1}}}
\def\PY@tok@s1{\def\PY@tc##1{\textcolor[rgb]{0.73,0.13,0.13}{##1}}}
\def\PY@tok@sh{\def\PY@tc##1{\textcolor[rgb]{0.73,0.13,0.13}{##1}}}
\def\PY@tok@sc{\def\PY@tc##1{\textcolor[rgb]{0.73,0.13,0.13}{##1}}}
\def\PY@tok@sx{\def\PY@tc##1{\textcolor[rgb]{0.00,0.50,0.00}{##1}}}
\def\PY@tok@bp{\def\PY@tc##1{\textcolor[rgb]{0.00,0.50,0.00}{##1}}}
\def\PY@tok@c1{\let\PY@it=\textit\def\PY@tc##1{\textcolor[rgb]{0.25,0.50,0.50}{##1}}}
\def\PY@tok@kc{\let\PY@bf=\textbf\def\PY@tc##1{\textcolor[rgb]{0.00,0.50,0.00}{##1}}}
\def\PY@tok@c{\let\PY@it=\textit\def\PY@tc##1{\textcolor[rgb]{0.25,0.50,0.50}{##1}}}
\def\PY@tok@mf{\def\PY@tc##1{\textcolor[rgb]{0.40,0.40,0.40}{##1}}}
\def\PY@tok@err{\def\PY@bc##1{\fcolorbox[rgb]{1.00,0.00,0.00}{1,1,1}{##1}}}
\def\PY@tok@kd{\let\PY@bf=\textbf\def\PY@tc##1{\textcolor[rgb]{0.00,0.50,0.00}{##1}}}
\def\PY@tok@ss{\def\PY@tc##1{\textcolor[rgb]{0.10,0.09,0.49}{##1}}}
\def\PY@tok@sr{\def\PY@tc##1{\textcolor[rgb]{0.73,0.40,0.53}{##1}}}
\def\PY@tok@mo{\def\PY@tc##1{\textcolor[rgb]{0.40,0.40,0.40}{##1}}}
\def\PY@tok@kn{\let\PY@bf=\textbf\def\PY@tc##1{\textcolor[rgb]{0.00,0.50,0.00}{##1}}}
\def\PY@tok@mi{\def\PY@tc##1{\textcolor[rgb]{0.40,0.40,0.40}{##1}}}
\def\PY@tok@gp{\let\PY@bf=\textbf\def\PY@tc##1{\textcolor[rgb]{0.00,0.00,0.50}{##1}}}
\def\PY@tok@o{\def\PY@tc##1{\textcolor[rgb]{0.40,0.40,0.40}{##1}}}
\def\PY@tok@kr{\let\PY@bf=\textbf\def\PY@tc##1{\textcolor[rgb]{0.00,0.50,0.00}{##1}}}
\def\PY@tok@s{\def\PY@tc##1{\textcolor[rgb]{0.73,0.13,0.13}{##1}}}
\def\PY@tok@kp{\def\PY@tc##1{\textcolor[rgb]{0.00,0.50,0.00}{##1}}}
\def\PY@tok@w{\def\PY@tc##1{\textcolor[rgb]{0.73,0.73,0.73}{##1}}}
\def\PY@tok@kt{\def\PY@tc##1{\textcolor[rgb]{0.69,0.00,0.25}{##1}}}
\def\PY@tok@ow{\let\PY@bf=\textbf\def\PY@tc##1{\textcolor[rgb]{0.67,0.13,1.00}{##1}}}
\def\PY@tok@sb{\def\PY@tc##1{\textcolor[rgb]{0.73,0.13,0.13}{##1}}}
\def\PY@tok@k{\let\PY@bf=\textbf\def\PY@tc##1{\textcolor[rgb]{0.00,0.50,0.00}{##1}}}
\def\PY@tok@se{\let\PY@bf=\textbf\def\PY@tc##1{\textcolor[rgb]{0.73,0.40,0.13}{##1}}}
\def\PY@tok@sd{\let\PY@it=\textit\def\PY@tc##1{\textcolor[rgb]{0.73,0.13,0.13}{##1}}}

\def\PYZbs{\char`\\}
\def\PYZus{\char`\_}
\def\PYZob{\char`\{}
\def\PYZcb{\char`\}}
\def\PYZca{\char`\^}
\def\PYZsh{\char`\#}
\def\PYZpc{\char`\%}
\def\PYZdl{\char`\$}
\def\PYZti{\char`\~}
% for compatibility with earlier versions
\def\PYZat{@}
\def\PYZlb{[}
\def\PYZrb{]}
\makeatother

\renewcommand{\footnotesize}{\tiny}


\begin{document}
\title{Semantische Modellierung von mathematischen Publikationen und Definition eines Ähnlichkeitsmaßes}
\author[Lyudmila Vaseva]{Lyudmila Vaseva\\{vaseva@mi.fu-berlin.de}}
%\email{vaseva@mi.fu-berlin.de}
\institute{Freie Universität Berlin \\
    Institut für Informatik}
\date{\today}

\begin{frame}
	\titlepage
\end{frame}

%Motivation! (ohne Folie)
\begin{frame}
    \begin{block}{Motivation}
%    \begin{itemize}
%
%    \end{itemize}
    \end{block}
    \begin{block}{Ziel}
    \begin{itemize}
    \item Präzisere Ähnlichkeitsberechnung für Publikationen aufgrund möglichst vieler Metadaten
    \end{itemize}
    \end{block}
\end{frame}

\begin{frame}
    \frametitle{Gliederung}
    \begin{enumerate}
        \item Existierende Repräsentationen
        \item Algorithmen zur Bestimmung der Ähnlichkeit von wissenschaftlichen Arbeiten
        \item Publikationsnetzwerk
        \item Semantische Ähnlichkeit
        \item Laufzeitanalyse
        \item Ergebnisse und Auswertung
    \end{enumerate}
\end{frame}

%Vlt weglassen
%\begin{frame}
%    \frametitle{Existierende Repräsentationen von Publikationen}
%\end{frame}

\begin{frame}
    \frametitle{Existierende Ähnlichkeitsmaße - textbasiert}
    Zum Beispiel das auf TF-IDF basierte Kosinusmaß
    \begin{block}{Nachteile}
    \begin{itemize}
        \item sprachabhängig
        \item nicht praktikabel für große Dokumente/Dokumentenmengen
        \item Semantik der Terme wird nicht erfasst
    \end{itemize}
    \end{block}
    %Vor- und Nachteile
\end{frame}

%Gleichungen?/Tabelle mit Überblick?
\begin{frame}
    \frametitle{Existierende Ähnlichkeitsmaße - zitationsbaseirt I}
\end{frame}

\begin{frame}
    \frametitle{Existierende Ähnlichkeitsmaße - zitationsbasiert II}
    \begin{block}{Weitere Beispiele}
    \begin{itemize}
        \item CC-IDF
        \item Lu et al.: Ähnlichkeit aufgrund lokaler Nachbarschaft
    \end{itemize}
    \end{block}
\end{frame}

\begin{frame}
    \frametitle{Existierende Ähnlichkeitsmaße - zitationsbasiert III}
    %Vor- und Nachteile
    \begin{block}{Nachteile}
    \begin{itemize}
        \item Gründe für das Zitieren unbekannt
        %spezifische Nachteile der einzelnen Ansätze
    \end{itemize}
    \end{block}
\end{frame}

\begin{frame}
    \frametitle{Publikationsnetzwerk}
    %Bild!
    \begin{center}
        \includegraphics[scale=0.38]{../deps/publications_ontology_relations.png}
    \end{center}
\end{frame}


\begin{frame}
    \frametitle{Semantische Ähnlichkeit}
    Formel!
\end{frame}

\begin{frame}
    \frametitle{Evaluation}
    \begin{itemize}
        \item Laufzeit: $\mathcal{O}(n^4)$
        \item Speicherkomplexität: $\mathcal{O}(n^2)$
        \item Der Testdatensatz enthält ursprünglich $2,907,086$ Publikationen
        \item Der Testdatensatz wird auf $3095$ Publikationen, $15,994$ Knoten insgesamt gekürzt
    \end{itemize}
\end{frame}

\begin{frame}
    \frametitle{Ergebnisse des semantischen Ähnlichkeitsmaßes}
    \begin{itemize}
        \item Vergleich mit \textit{C-Rank}
    \end{itemize}
\end{frame}

\begin{frame}
    \frametitle{Fazit}
    \begin{itemize}
    \item Die semantische Ähnlichkeit liefert bessere Ergebnisse als Ansätze, die nur auf einer Art Metadaten basieren
    \item Semantische Ähnlichkeit sollte für das vollständige Publikationsnetzwerk berechnet werden
    \item Der Ansatz ist leider nicht praktikabel für große Datenmengen
        \begin{itemize}
            \item Laufzeitoptimierung durch Parallelisierung?
            \item Laufzeitoptimierung durch nicht-iterative Berechnung?
            \item Speicheroptimierung durch Auslagern der Daten in eine Datenbank
        \end{itemize}
    \end{itemize}
\end{frame}

\begin{frame}
\centerline{Vielen Dank!}
\end{frame}

%\begin{frame}
%	\frametitle{Eigenschaften des Datensatzes}
%	\begin{itemize}
%		\item  enthält ca. $706\,000$ Einträge
%		\item  mit 19 verschiedenen Themengebieten
%		\item  nur der Themenbereich Physik wird in Themengruppen unterteilt
%		\item  11 Einträge ohne Informationen
%		\item  Publikationen haben im Durchschnitt 1.3 und maximal 9 Themen
%	\end{itemize}
%\end{frame}

\end{document}
