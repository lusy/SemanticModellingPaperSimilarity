\documentclass[12pt, xcolor=table]{beamer}
\usepackage{graphicx}
\usepackage[ngerman]{babel}
\usepackage[utf8]{inputenc}
\usepackage{amsmath}
\usepackage{amssymb}
\usepackage{listings}
\usepackage{hyperref}
\usepackage{fancyvrb}
\usepackage{color}

\usepackage[percent]{overpic}
\usepackage[footnotesize, bf]{caption}
\input{theme.tex}
\input{syntax}
\renewcommand{\footnotesize}{\tiny}


\begin{document}
\title{Semantische Modellierung von mathematischen Publikationen und Definition eines Ähnlichkeitsmaßes}
\author[Lyudmila Vaseva]{Lyudmila Vaseva\\{vaseva@mi.fu-berlin.de}}
%\email{vaseva@mi.fu-berlin.de}
\institute{Freie Universität Berlin \\
    Institut für Informatik}
\date{\today}

\begin{frame}
	\titlepage
\end{frame}

\begin{frame}
	\frametitle{Eigenschaften des Datensatzes}
	\begin{itemize}
		\item  enthält ca. $706\,000$ Einträge
		\item  mit 19 verschiedenen Themengebieten
		\item  nur der Themenbereich Physik wird in Themengruppen unterteilt
		\item  11 Einträge ohne Informationen
		\item  Publikationen haben im Durchschnitt 1.3 und maximal 9 Themen
	\end{itemize}
\end{frame}
\end{document}
