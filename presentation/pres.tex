\documentclass[12pt, xcolor=table]{beamer}
\usepackage{graphicx}
\usepackage[ngerman]{babel}
\usepackage[utf8]{inputenc}
\usepackage{amsmath}
\usepackage{amssymb}
\usepackage{listings}
\usepackage{hyperref}
\usepackage{fancyvrb}
\usepackage{color}
\usepackage{comment}

\usepackage[percent]{overpic}
\usepackage[footnotesize, bf]{caption}
%Copyright 2008 by Adrian Böhmichen
%
% This file is free software: you can redistribute it and/or modify
% it under the terms of the GNU General Public License as published by
% the Free Software Foundation, either version 3 of the License, or
% (at your option) any later version.
%
% This file is distributed in the hope that it will be useful,
% but WITHOUT ANY WARRANTY; without even the implied warranty of
% MERCHANTABILITY or FITNESS FOR A PARTICULAR PURPOSE.  See the
% GNU General Public License for more details.
%
% You should have received a copy of the GNU General Public License
% along with this file.  If not, see <http://www.gnu.org/licenses/>.

%%%%%%%%%%%%%%%%%%%%%%%%%%%%%%%%%%%%%%%%%%%%%%%%%%%%%%%%%%%%%%%%%
%     Ubuntuusers Vorlage für ein LaTeX-Beamer Theme            %
%                                                               %
% Für das Korrekte funktionieren benötigt man einen header.png  %
% und ein logo.png Datei!                                       %
% Zusätzlich muss man folgende Pakete benutzten:                %
%   \usepackage{graphicx}                                       %
%   \usepackage[percent]{overpic}                               %
%                                                               %
% Danach muss nur noch am Anfang die Datei                      %
% mit \input{} eingebunden werden.                              %
%                                                               %
%%%%%%%%%%%%%%%%%%%%%%%%%%%%%%%%%%%%%%%%%%%%%%%%%%%%%%%%%%%%%%%%%

%weitere Farbe spezifizieren:
%Farben von dem Humantheme
%\definecolor{Orange}{RGB}{240,165,19}
\definecolor{Orange}{RGB}{5,215,242}
%\definecolor{Human-Base}{RGB}{129,102,71}
\definecolor{Human-Base}{RGB}{5,25,242}
%Farben aus dem Inyokatheme
%\definecolor{uuheader1}{RGB}{164,143,101}
\definecolor{uuheader1}{RGB}{5,25,242}
%\definecolor{uuheader2}{RGB}{129,106,59}
\definecolor{uuheader2}{RGB}{5,25,242}


%Theme festlegen für alle Templates die nicht selbstständig definiert werden:
\usepackage{beamerthemedefault}


%Definieren des Innertheme, zuständig für die Symbole bei Listen
\setbeamertemplate{sections/subsections in toc}[square]
\setbeamertemplate{items}[circle]

\setbeamercolor{item}{fg=Human-Base}

%entfernen der Navigationsleiste
\beamertemplatenavigationsymbolsempty

%Logo definieren, man kann die Lage nicht verändern
%\logo{\includegraphics[scale=0.1]{logo.png}}


%Kopf- und Fußzeile definieren
%\setbeamertemplate{headline}
%{%
%\begin{overpic}[width=\paperwidth
% nächste Zeile dient zum anzeigen eines Rasters, für das paltzieren des ToC hilfreich
%,grid,tics=10
%]
%{header.png}%
%  \put(0,11){\insertsectionnavigationhorizontal{\paperwidth}{~}{~}}%
%  \end{overpic}
%}

\setbeamertemplate{footline}[text line]
{%
\begin{minipage}[b]{116mm}
\insertauthor \hfill%
%neue Navigationsleiste
 \insertframenumber ~/ \inserttotalframenumber\\[1ex]
\end{minipage}
}

% Farben festlegen ausserhalb des innertheme

%Allgemeine Angaben und Verbesserung vom default Theme
\setbeamercolor{structure}{fg=uuheader1}
\setbeamercolor{section in toc}{fg=Human-Base}
\setbeamercolor{subsection in toc}{parent=section in toc}
\setbeamercolor{framesubtitle}{fg=uuheader2}


%Farbe und Form der Blöcke definieren
\setbeamertemplate{blocks}[rounded]
%\setbeamercolor{block title}{fg=uuheader1,bg=Orange}
%\setbeamercolor{block title alerted}{use=alerted text,fg=black,bg=alerted text.fg!75!bg}
%\setbeamercolor{block title example}{use=example text,fg=black,bg=example text.fg!75!bg}

%\setbeamercolor{block body}{parent=normal text,use=block title,bg=block title.bg!25!bg}
%\setbeamercolor{block body alerted}{parent=normal text,use=block title alerted,bg=block title alerted.bg!25!bg}
%\setbeamercolor{block body example}{parent=normal text,use=block title example,bg=block title example.bg!25!bg}

%Für den Titleframe
\setbeamertemplate{title page}[default][rounded=true]
\setbeamercolor{title}{fg=uuheader2,bg=Orange}


\makeatletter
\def\PY@reset{\let\PY@it=\relax \let\PY@bf=\relax%
    \let\PY@ul=\relax \let\PY@tc=\relax%
    \let\PY@bc=\relax \let\PY@ff=\relax}
\def\PY@tok#1{\csname PY@tok@#1\endcsname}
\def\PY@toks#1+{\ifx\relax#1\empty\else%
    \PY@tok{#1}\expandafter\PY@toks\fi}
\def\PY@do#1{\PY@bc{\PY@tc{\PY@ul{%
    \PY@it{\PY@bf{\PY@ff{#1}}}}}}}
\def\PY#1#2{\PY@reset\PY@toks#1+\relax+\PY@do{#2}}

\def\PY@tok@gd{\def\PY@tc##1{\textcolor[rgb]{0.63,0.00,0.00}{##1}}}
\def\PY@tok@gu{\let\PY@bf=\textbf\def\PY@tc##1{\textcolor[rgb]{0.50,0.00,0.50}{##1}}}
\def\PY@tok@gt{\def\PY@tc##1{\textcolor[rgb]{0.00,0.25,0.82}{##1}}}
\def\PY@tok@gs{\let\PY@bf=\textbf}
\def\PY@tok@gr{\def\PY@tc##1{\textcolor[rgb]{1.00,0.00,0.00}{##1}}}
\def\PY@tok@cm{\let\PY@it=\textit\def\PY@tc##1{\textcolor[rgb]{0.25,0.50,0.50}{##1}}}
\def\PY@tok@vg{\def\PY@tc##1{\textcolor[rgb]{0.10,0.09,0.49}{##1}}}
\def\PY@tok@m{\def\PY@tc##1{\textcolor[rgb]{0.40,0.40,0.40}{##1}}}
\def\PY@tok@mh{\def\PY@tc##1{\textcolor[rgb]{0.40,0.40,0.40}{##1}}}
\def\PY@tok@go{\def\PY@tc##1{\textcolor[rgb]{0.50,0.50,0.50}{##1}}}
\def\PY@tok@ge{\let\PY@it=\textit}
\def\PY@tok@vc{\def\PY@tc##1{\textcolor[rgb]{0.10,0.09,0.49}{##1}}}
\def\PY@tok@il{\def\PY@tc##1{\textcolor[rgb]{0.40,0.40,0.40}{##1}}}
\def\PY@tok@cs{\let\PY@it=\textit\def\PY@tc##1{\textcolor[rgb]{0.25,0.50,0.50}{##1}}}
\def\PY@tok@cp{\def\PY@tc##1{\textcolor[rgb]{0.74,0.48,0.00}{##1}}}
\def\PY@tok@gi{\def\PY@tc##1{\textcolor[rgb]{0.00,0.63,0.00}{##1}}}
\def\PY@tok@gh{\let\PY@bf=\textbf\def\PY@tc##1{\textcolor[rgb]{0.00,0.00,0.50}{##1}}}
\def\PY@tok@ni{\let\PY@bf=\textbf\def\PY@tc##1{\textcolor[rgb]{0.60,0.60,0.60}{##1}}}
\def\PY@tok@nl{\def\PY@tc##1{\textcolor[rgb]{0.63,0.63,0.00}{##1}}}
\def\PY@tok@nn{\let\PY@bf=\textbf\def\PY@tc##1{\textcolor[rgb]{0.00,0.00,1.00}{##1}}}
\def\PY@tok@no{\def\PY@tc##1{\textcolor[rgb]{0.53,0.00,0.00}{##1}}}
\def\PY@tok@na{\def\PY@tc##1{\textcolor[rgb]{0.49,0.56,0.16}{##1}}}
\def\PY@tok@nb{\def\PY@tc##1{\textcolor[rgb]{0.00,0.50,0.00}{##1}}}
\def\PY@tok@nc{\let\PY@bf=\textbf\def\PY@tc##1{\textcolor[rgb]{0.00,0.00,1.00}{##1}}}
\def\PY@tok@nd{\def\PY@tc##1{\textcolor[rgb]{0.67,0.13,1.00}{##1}}}
\def\PY@tok@ne{\let\PY@bf=\textbf\def\PY@tc##1{\textcolor[rgb]{0.82,0.25,0.23}{##1}}}
\def\PY@tok@nf{\def\PY@tc##1{\textcolor[rgb]{0.00,0.00,1.00}{##1}}}
\def\PY@tok@si{\let\PY@bf=\textbf\def\PY@tc##1{\textcolor[rgb]{0.73,0.40,0.53}{##1}}}
\def\PY@tok@s2{\def\PY@tc##1{\textcolor[rgb]{0.73,0.13,0.13}{##1}}}
\def\PY@tok@vi{\def\PY@tc##1{\textcolor[rgb]{0.10,0.09,0.49}{##1}}}
\def\PY@tok@nt{\let\PY@bf=\textbf\def\PY@tc##1{\textcolor[rgb]{0.00,0.50,0.00}{##1}}}
\def\PY@tok@nv{\def\PY@tc##1{\textcolor[rgb]{0.10,0.09,0.49}{##1}}}
\def\PY@tok@s1{\def\PY@tc##1{\textcolor[rgb]{0.73,0.13,0.13}{##1}}}
\def\PY@tok@sh{\def\PY@tc##1{\textcolor[rgb]{0.73,0.13,0.13}{##1}}}
\def\PY@tok@sc{\def\PY@tc##1{\textcolor[rgb]{0.73,0.13,0.13}{##1}}}
\def\PY@tok@sx{\def\PY@tc##1{\textcolor[rgb]{0.00,0.50,0.00}{##1}}}
\def\PY@tok@bp{\def\PY@tc##1{\textcolor[rgb]{0.00,0.50,0.00}{##1}}}
\def\PY@tok@c1{\let\PY@it=\textit\def\PY@tc##1{\textcolor[rgb]{0.25,0.50,0.50}{##1}}}
\def\PY@tok@kc{\let\PY@bf=\textbf\def\PY@tc##1{\textcolor[rgb]{0.00,0.50,0.00}{##1}}}
\def\PY@tok@c{\let\PY@it=\textit\def\PY@tc##1{\textcolor[rgb]{0.25,0.50,0.50}{##1}}}
\def\PY@tok@mf{\def\PY@tc##1{\textcolor[rgb]{0.40,0.40,0.40}{##1}}}
\def\PY@tok@err{\def\PY@bc##1{\fcolorbox[rgb]{1.00,0.00,0.00}{1,1,1}{##1}}}
\def\PY@tok@kd{\let\PY@bf=\textbf\def\PY@tc##1{\textcolor[rgb]{0.00,0.50,0.00}{##1}}}
\def\PY@tok@ss{\def\PY@tc##1{\textcolor[rgb]{0.10,0.09,0.49}{##1}}}
\def\PY@tok@sr{\def\PY@tc##1{\textcolor[rgb]{0.73,0.40,0.53}{##1}}}
\def\PY@tok@mo{\def\PY@tc##1{\textcolor[rgb]{0.40,0.40,0.40}{##1}}}
\def\PY@tok@kn{\let\PY@bf=\textbf\def\PY@tc##1{\textcolor[rgb]{0.00,0.50,0.00}{##1}}}
\def\PY@tok@mi{\def\PY@tc##1{\textcolor[rgb]{0.40,0.40,0.40}{##1}}}
\def\PY@tok@gp{\let\PY@bf=\textbf\def\PY@tc##1{\textcolor[rgb]{0.00,0.00,0.50}{##1}}}
\def\PY@tok@o{\def\PY@tc##1{\textcolor[rgb]{0.40,0.40,0.40}{##1}}}
\def\PY@tok@kr{\let\PY@bf=\textbf\def\PY@tc##1{\textcolor[rgb]{0.00,0.50,0.00}{##1}}}
\def\PY@tok@s{\def\PY@tc##1{\textcolor[rgb]{0.73,0.13,0.13}{##1}}}
\def\PY@tok@kp{\def\PY@tc##1{\textcolor[rgb]{0.00,0.50,0.00}{##1}}}
\def\PY@tok@w{\def\PY@tc##1{\textcolor[rgb]{0.73,0.73,0.73}{##1}}}
\def\PY@tok@kt{\def\PY@tc##1{\textcolor[rgb]{0.69,0.00,0.25}{##1}}}
\def\PY@tok@ow{\let\PY@bf=\textbf\def\PY@tc##1{\textcolor[rgb]{0.67,0.13,1.00}{##1}}}
\def\PY@tok@sb{\def\PY@tc##1{\textcolor[rgb]{0.73,0.13,0.13}{##1}}}
\def\PY@tok@k{\let\PY@bf=\textbf\def\PY@tc##1{\textcolor[rgb]{0.00,0.50,0.00}{##1}}}
\def\PY@tok@se{\let\PY@bf=\textbf\def\PY@tc##1{\textcolor[rgb]{0.73,0.40,0.13}{##1}}}
\def\PY@tok@sd{\let\PY@it=\textit\def\PY@tc##1{\textcolor[rgb]{0.73,0.13,0.13}{##1}}}

\def\PYZbs{\char`\\}
\def\PYZus{\char`\_}
\def\PYZob{\char`\{}
\def\PYZcb{\char`\}}
\def\PYZca{\char`\^}
\def\PYZsh{\char`\#}
\def\PYZpc{\char`\%}
\def\PYZdl{\char`\$}
\def\PYZti{\char`\~}
% for compatibility with earlier versions
\def\PYZat{@}
\def\PYZlb{[}
\def\PYZrb{]}
\makeatother

\renewcommand{\footnotesize}{\tiny}


\begin{document}
\title{Semantische Modellierung von mathematischen Publikationen und Definition eines Ähnlichkeitsmaßes}
\author[Lyudmila Vaseva]{Lyudmila Vaseva\\{vaseva@mi.fu-berlin.de}}
%\email{vaseva@mi.fu-berlin.de}
\institute{Freie Universität Berlin \\
    Institut für Informatik}
\date{\today}

\begin{frame}
	\titlepage
\end{frame}


\begin{frame}
    \frametitle{Gliederung}
    \begin{enumerate}
        \item Motivation
        \item Existierende Algorithmen zur Bestimmung der Ähnlichkeit von wissenschaftlichen Arbeiten
        \item Publikationsnetzwerk
        \item Semantische Ähnlichkeit
        \item Semantische Ähnlichkeit: Analyse, Ergebnisse und Auswertung
        \item Fazit
    \end{enumerate}
\end{frame}

%Vlt weglassen
%\begin{frame}
%    \frametitle{Existierende Repräsentationen von Publikationen}
%\end{frame}

%Motivation! (ohne Folie)
\begin{frame}
    \begin{block}{Motivation}
    \begin{itemize}
        \item Recommender Systeme
        \item Automatisierte Klassifikation
    \end{itemize}
    \end{block}

    \begin{block}{Ziel}
    \begin{itemize}
    \item Präzisere Ähnlichkeitsberechnung für Publikationen aufgrund möglichst vieler Metadaten
    \end{itemize}
    \end{block}
\end{frame}

\begin{frame}
    \frametitle{Existierende Ähnlichkeitsmaße}
    \begin{columns}[T]
        \begin{column}{0.45\textwidth}
        \begin{block}{textbasiert}
            \begin{itemize}
                \item sprachabhängig
                \item unbekannte Semantik der Terme
                \item nicht praktikabel bei großen Datensätzen
            \end{itemize}
        \end{block}
        \end{column}

        \begin{column}{0.45\textwidth}
        \begin{block}{zitationsbasiert}
            \begin{itemize}
                \item Grund für Zitieren unbekannt
            \end{itemize}
        \end{block}
        \end{column}
    \end{columns}

%    \begin{table}[h]
%   \centering
%    \begin{tabular}{p{6cm} p{6cm}}
%    textbasiert & zitationsbasiert\\
%    \hline
%    sprachabhängig & Grund für Zitieren unbekannt\\
%    unbekannte Semantik der Terme & \\
%    nicht praktikabel bei großen Datensätzen\\
%\end{tabular}
%\end{table}
\end{frame}

\begin{comment}
\begin{frame}
    \frametitle{Existierende Ähnlichkeitsmaße - textbasiert}
    Beispiel: Kosinusmaß basierend auf TF-IDF
    \begin{block}{Nachteile}
    \begin{itemize}
        \item sprachabhängig
        \item nicht praktikabel für große Dokumente/Dokumentenmengen
        \item Semantik der Terme wird nicht erfasst
    \end{itemize}
    \end{block}
    %Vor- und Nachteile
\end{frame}


%Gleichungen?/Tabelle mit Überblick?
\begin{frame}
    \frametitle{Existierende Ähnlichkeitsmaße - zitationsbaseirt I}
    \begin{block}{\small{Bibiliographische Kopplung}}
    \small{
        $sim(a,b) = O(a) \cap O(b)$
    }
    \end{block}

    \begin{block}{\small{Kozitation}}
    \small{
         $sim(a,b) = I(a) \cap I(b)$
    }
    \end{block}

    \begin{block}{\small{Amsler}}
    \small{
        $sim(a,b) = \cfrac{L(a) \cap L(b)}{L(a) \cup L(b)}$
    }
    \end{block}
\end{frame}

\begin{frame}
    \frametitle{Existierende Ähnlichkeitsmaße - zitationsbasiert II}
    \begin{block}{\small{simRank}}
    \scriptsize{
        $sim(a,b) = 1 \quad \text{wenn } a=b$
        \newline
        sonst: \quad
        $sim(a,b)  = 
        \cfrac{c}{|I(a)||I(b)|}
        \sum_{i=1}^{|I(a)|} \sum_{j=1}^{|I(b)|} sim(I_i(a),I_j(b))$
    }
    \end{block}

    \begin{block}{\small{P-Rank}}
    \scriptsize{
        $sim(a,b) = 1 \quad \text{wenn } a=b$
        \newline
        sonst: \quad
        $sim(a,b)  = 
        \lambda\times\cfrac{c}{|I(a)||I(b)|}
        \sum_{i=1}^{|I(a)|} \sum_{j=1}^{|I(b)|} sim(I_i(a),I_j(b))
         +
        (1-\lambda)\times\cfrac{c}{|O(a)||O(b)|}
        \sum_{i=1}^{|O(a)|} \sum_{j=1}^{|O(b)|} sim(O_i(a),O_j(b))$
    }
    \end{block}

    \begin{block}{\small{C-Rank}}
    \scriptsize{
        $sim(a,b) = 1 \quad \text{wenn } a=b$
        \newline
        sonst: \quad
        $sim(a,b)  = 
        \cfrac{c}{|L(a)||L(b)|}
        \sum_{i=1}^{|L(a)|} \sum_{j=1}^{|L(b)|} sim(L_i(a),L_j(b))$
   }
    \end{block}

\end{frame}


\begin{frame}
    \frametitle{Existierende Ähnlichkeitsmaße - zitationsbasiert III}
    \begin{block}{Weitere Beispiele}
    \begin{itemize}
        \item CC-IDF
        \item Lu et al.: Ähnlichkeit aufgrund lokaler Nachbarschaft
    \end{itemize}
    \end{block}
\end{frame}

\begin{frame}
    \frametitle{Existierende Ähnlichkeitsmaße - zitationsbasiert IV}
    %Vor- und Nachteile
    \begin{block}{Nachteile}
    \begin{itemize}
        \item Gründe für das Zitieren unbekannt
        %spezifische Nachteile der einzelnen Ansätze
    \end{itemize}
    \end{block}
\end{frame}
\end{comment}

\begin{frame}
    \frametitle{Publikationsnetzwerk}
    %Bild!
    \begin{center}
        \includegraphics[scale=0.38]{../deps/publications_ontology_relations.png}
    \end{center}
\end{frame}


\begin{frame}
    \frametitle{Semantische Ähnlichkeit}
    %Formel!
    %Global auf das Netzwerk
    %Rekursive Natur
    %kann/wird iterativ berechnet, konvergiert zum Fixpunkt
    %Für jedes Elementenpaar: schaue alle Elementenpaare, die sich aus den Nachbarn bilden an -- Laufzeit
    %Verschiedenartige Metadaten werden durch Parameter gewichtet
    %Dämpfungskonstante, die bestimmt wie stark weiter entfernte Knoten die aktuellen Knoten beeinflussen
\scriptsize{
\[
 R_0(a,b) =
    \begin{cases}
     1 & \text{wenn } $a$ = $b$ \\
     0 & \text{sonst}\\
    \end{cases}
\]
\newline
\text{wenn a und b Publikationen:}
\[
\begin{array}{lcl}
 R_{k+1}(a,b) & = & 
        \lambda_1\times\cfrac{c}{|K(a)||K(b)|}
        \sum_{i=1}^{|K(a)|} \sum_{j=1}^{|K(b)|} R_k(K_i(a),K_j(b))
        \\ & + &
        \lambda_2\times\cfrac{c}{|A(a)||A(b)|}
        \sum_{i=1}^{|A(a)|} \sum_{j=1}^{|A(b)|} R_k(A_i(a),A_j(b))
        \\ & + &
        \lambda_3\times\cfrac{c}{|S(a)||S(b)|}
        \sum_{i=1}^{|S(a)|} \sum_{j=1}^{|S(b)|} R_k(S_i(a),S_j(b))
        \\ & + &
        \lambda_4\times\cfrac{c}{|C(a)||C(b)|}
        \sum_{i=1}^{|C(a)|} \sum_{j=1}^{|C(b)|} R_k(C_i(a),C_j(b))
        \\ & + &
        \lambda_5\times\cfrac{c}{|Y(a)||Y(b)|}
        \sum_{i=1}^{|Y(a)|} \sum_{j=1}^{|Y(b)|} R_k(Y_i(a),Y_j(b))
\end{array}
\]
\newline
\text{wenn a und b aus sonstiger Klasse:}
\[
R_{k+1}(a,b)  = 
        \cfrac{c}{|L(a)||L(b)|}
        \sum_{i=1}^{|L(a)|} \sum_{j=1}^{|L(b)|} R_k(L_i(a),L_j(b))
\]
}
\end{frame}

\begin{frame}
    \frametitle{Evaluation}
    \begin{itemize}
        \item Laufzeit: $\mathcal{O}(n^4)$
        \item Speicherkomplexität: $\mathcal{O}(n^2)$
        \item Der Testdatensatz enthält ursprünglich $2,907,086$ Publikationen
        \item Der Testdatensatz wird auf $3095$ Publikationen, $15,994$ Knoten insgesamt gekürzt - alle Publikationen bis 1975
    \end{itemize}
\end{frame}

\begin{frame}
    \frametitle{Ergebnisse des semantischen Ähnlichkeitsmaßes}
    \scriptsize{
    \begin{table}[H]
    \centering
    \begin{tabular}{l c c c}
		%\hline
		\textbf{Parameter} &\textbf{Gewichtung 1} &\textbf{Gewichtung 2} &\textbf{Gewichtung 3} \\
		%\hline
		$\lambda_1$, Keywords & $0.5$ & $0.2$ & $0.4$\\
		$\lambda_2$, Autoren & $0.1$ & $0.2$ & $0.3$\\
		$\lambda_3$, Quellen & $0.1$ & $0.2$ & $0.1$\\
		$\lambda_4$, Zitationen & $0.2$ & $0.2$ & $0.1$\\
		$\lambda_5$, Erscheinungsjahre & $0.1$ & $0.2$ & $0.1$\\
	    %\hline
    \end{tabular}
    \end{table}
    }
    \scriptsize{
    \begin{table}[H]
    \centering
    \begin{tabular}{l l l l l}
	    	%\hline
		     &\textbf{Gewichtung 1} &\textbf{Gewichtung 2} &\textbf{Gewichtung 3} & \textbf{C-Rank} \\
    		%\hline
	    	Entropy & $2.4863$ & $3.28742$ & $2.48117$ & $4.13098$\\
		    Purity & $0.50953$ & $0.39774$ & $0.51599$ & $0.35767$\\
    		Silhouette & $0.084$ & $0.056$ & $0.069$ & $0.0$ \\
	        %\hline
    \end{tabular}
\end{table}
}
%   \begin{itemize}
%        \item Vergleich mit \textit{C-Rank}
%    \end{itemize}
\end{frame}

\begin{frame}
    \frametitle{Fazit}
    \begin{itemize}
    \item bessere Ergebnisse als Maße, die nur auf einer Art Metadaten basieren
    \item Datensatzverkürzung beeinträchtigt die Ergebnisse
    \item nicht praktikabel für große Datenmengen
        \begin{itemize}
            \item Laufzeitoptimierung durch Parallelisierung?
            \item Laufzeitoptimierung durch nicht-iterative Berechnung?
            \item Speicheroptimierung durch Auslagern der Daten in eine Datenbank
        \end{itemize}
    \end{itemize}
\end{frame}

\begin{frame}
\centerline{Vielen Dank!}
\end{frame}

\end{document}
